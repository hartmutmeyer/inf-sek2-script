\clearpage

\lehead[]{\sf\hspace*{-2.00cm}\textcolor{white}{\colorbox{lightblue}{\parbox[c][0.70cm][b]{1.60cm}{
\makebox[1.60cm][r]{\thechapter}\\ \makebox[1.60cm][r]{ÜBUNG}}}}\hspace{0.17cm}\textcolor{lightblue}{\chaptertitle}}
\rohead[]{\textcolor{lightblue}{\chaptertitle}\sf\hspace*{0.17cm}\textcolor{white}{\colorbox{lightblue}{\parbox[c][0.70cm][b]{1.60cm}{\thechapter\\
ÜBUNG}}}\hspace{-2.00cm}}
%\chead[]{}
\rehead[]{\textcolor{lightblue}{AvHG, Inf, My}}
\lohead[]{\textcolor{lightblue}{AvHG, Inf, My}}

\section{Hardware und Software -- Übungen}

\subsection{Aufgabe 1: Speicherarten}

\begin{compactenum}[a)]
\item Was versteht man unter \glqq flüchtigem\grqq\ Speicher?
\item Was ist der Unterschied zwischen dem Hauptspeicher und dem RAM?
\item Erkläre die unterschiedlichen Aufgaben von Hauptspeicher und Festplatte.
\item Wozu dient der Cache?
\item Fülle die folgende Tabelle aus:
\end{compactenum}

\bgroup
\def\arraystretch{1.2}
\begin{tabular}{|l|c|c|c|c|c|}
\hline
 & \textbf{Register} & \textbf{Cache} & \textbf{RAM} & \textbf{Festplatte} &
 \textbf{USB-Stick} \\ \hline
\textbf{flüchtiger Speicher (ja/nein)} & & & & & \\ \hline
\textbf{Zugriffsgeschwindigkeit} & & & & & \\ \hline
\textbf{übliche Speichergröße} & & & & & \\ \hline
\end{tabular}
\egroup

\subsection{Aufgabe 2: Software}

\begin{compactenum}[a)]
\item Welcher Teil der Software liegt nicht auf der Festplatte, sondern
befindet sich auf einem ROM-Baustein (engl.\ \emph{Read Only Memory},
schreibgeschützter Speicher) auf der Hauptplatine (Mainboard)?

\item Wie wird ein Betriebssystem hergestellt?

\item Nenne die Namen einiger dir bekannter Betriebssysteme.

\item Nenne die Namen einiger dir bekannter Anwendungsprogramme.

\item Welche Aufgaben hat ein Betriebssystem?

\item Ein Benutzer führt mit dem Textverarbeitungsprogramm Word die
nachfolgenden Aktionen durch. Gib an, bei welchen der Aktionen neben dem
Textverarbeitungsprogramm auch das Betriebssystem oder ein Gerätetreiber in
Aktion treten muss.

\begin{compactenum}[1.]
\item Starten des Programms Word bzw.\ LibreOffice Writer.
\item Öffnen der bereits vorhandenen Datei \myFile{MeinText.doc}.
\item Veränderung der Schriftart des Textes von \glqq Arial\grqq\ nach
\glqq Times New Roman\grqq .
\item Abspeichern des geänderten Textes.
\item Ausgabe des Textes auf den Drucker.
\end{compactenum}

\item Begründe, wieso das Betriebssystem in residente und transiente Bestandteile
 unterteilt wird.
 
\item Wie kann man auf einem PC die Einstellungen des BIOS verändern?
\end{compactenum}


\subsection{Aufgabe 3: Stärken und Schwächen von MS Windows}

Die unten stehende Anekdote zeigt die Schwachstellen der heutigen
Betriebssysteme auf, die inzwischen meist völlig unübersichtlich und voller
Fehler sind.

1997 soll Bill Gates, der damalige Kopf des Microsoft-Konzerns, auf einer großen
Computer-Messe (COMDEX) die Computer-Industrie mit der Auto-Industrie
verglichen und das folgende Statement abgegeben haben:

\begin{quotation}
\noindent If GM had kept up with technology like the computer industry has, we
would all be driving \$25 cars that got 1.000 MPG.\footnote{Das Zitat
stammt so nicht tatsächlich von Bill Gates. Vielmer wurde eine echte Aussage
von ihm im nachhinein zugespitzt. Das echte Zitat ist: \glqq The PC industry is
different than any other industry. The volume, the openness, the innovation,
it's really unequaled. In fact, comparisons are often done between this
industry and others, and it's just stunning when you look at it. The price of a
mid-sized auto, it's about double what it used to be. Cereal, I admit I don't
buy that much cereal, but research shows that, too, has doubled in price. And
if you take that and say, what would those prices be if it were like the PC
industry, the car would cost about \$27, and the cereal would cost about one
cent. So, I think there's a lot to be learned by watching how this industry has
done what it's done.\grqq}
\end{quotation}

%“Wenn General Motors (GM) mit der Technologie so mitgehalten hätte, wie die
% Computer Industrie, dann würden wir heute alle 25-Dollar-Autos fahren, die
% 1000 Meilen pro Gallone Sprit fahren würden."

Als Antwort darauf veröffentlichte General Motors eine Presse-Erklärung mit
folgendem Inhalt:

\begin{quotation}
\noindent If GM had developed technology like Microsoft, we would all be driving
cars with the following characteristics: 

\begin{compactenum}
\item For no reason at all, your car would crash twice a day.
\item Every time they repainted the lines on the road, you would have to buy a
new car.
\item Occasionally, executing a manoeuver such as a left-turn would cause your
car to shut down and refuse to restart, and you would have to reinstall the
engine.
\item When your car died on the freeway for no reason, you would just accept
this, restart and drive on.
\item Only one person at a time could use the car, unless you bought 'Car95' or
'CarNT', and then added more seats.
\item Apple would make a car powered by the sun, reliable, five times as fast,
and twice as easy to drive, but would run on only five per cent of the roads.
\item Oil, water temperature and alternator warning lights would be replaced by
a single 'general car default' warning light.
\item New seats would force every-one to have the same size butt.
\item The airbag would say 'Are you sure?' before going off.
\item Occasionally, for no reason, your car would lock you out and refuse to let
you in until you simultaneously lifted the door handle, turned the key, and
grabbed the radio antenna.
\item GM would require all car buyers to also purchase a deluxe set of road maps
from Rand-McNally (a subsidiary of GM), even though they neither need them nor
want them. Trying to delete this option would immediately cause the car's
performance to diminish by 50 per cent or more. Moreover, GM would become a
target for investigation by the Justice Department.
\item Every time GM introduced a new model, car buyers would have to learn how
to drive all over again because none of the controls would operate in the same
manner as the old car.
\item You would press the 'start' button to shut off the engine.
\end{compactenum}
\end{quotation}

%"Wenn General Motors eine Technologie wie Microsoft entwickelt hätte, dann
% würden wir heute alle Autos mit folgenden Eigenschaften fahren:
%1. Ihr Auto würde ohne erkennbaren Grund zweimal am Tag einen Unfall haben.
%2. Jedes mal, wenn die Linien auf der Straße neu gezeichnet würden, müsste man
% ein neues Auto kaufen.
%3. Gelegentlich würde sich ein Auto ohne erkennbaren Grund auf der Autobahn
% einfach abstellen und man würde das einfach akzeptieren, neu starten und weiterfahren.
%4. Wenn man bestimmte Manöver durchführt, wie z.B. eine Linkskurve, würde sich
% das Auto einfach abstellen und sich weigern, neu zu starten. Man müsste dann den Motor erneut installieren.
%5. Man kann nur alleine in dem Auto sitzen, es sei denn, man kauft "Car95" oder
% "CarNT". Aber dann müsste man jeden Sitz einzeln bezahlen.
%6. Macintosh würde Autos herstellen, die mit Sonnenenergie fahren, zuverlässig
% laufen, fünfmal so schnell und zweimal so leicht zu fahren sind, aber sie laufen nur auf 5% der Straßen.
%7. Die Öl-Kontroll-Leuchte, die Warnlampen für Temperatur und Batterie würden
% durch eine "Genereller Auto-Fehler" Warnlampe ersetzt.
%8. Neue Sitze würden erfordern, dass alle dieselbe Gesäß-Größe haben.
%9. Das Airbag-System würde fragen "Sind sie sicher ?" bevor es auslöst.
%10. Gelegentlich würde das Auto sie ohne jeden erkennbaren Grund aussperren.
% Sie können nur wieder mit einem Trick aufschließen, und zwar müsste man gleichzeitig den Türgriff ziehen, den
%Schlüssel drehen und mit einer Hand an die Radioantenne fassen.
%11. General Motors würde Sie zwingen, mit jedem Auto einen Deluxe Kartensatz
% der Firma Rand McNally (seit neuestem eine GM Tochter) mit zu kaufen, auch wenn Sie diesen Kartensatz nicht
%brauchen oder möchten. Wenn Sie diese Option nicht wahrnehmen, würde das Auto
% sofort 50%
%langsamer werden (oder schlimmer). Darüber hinaus würde GM deswegen ein Ziel
% von Untersuchungen der Justiz.
%12. Immer dann, wenn ein neues Auto von GM vorgestellt werden würde, müssten
% alle Autofahrer das Autofahren neu erlernen, weil keiner der Bedienungs-Hebel genau so funktionieren würde, wie in den
%alten Autos.
%13. Man müsste den "Start"-Knopf drücken, um den Motor auszuschalten."


Aufgabe:

\begin{compactenum}
\item Kommentiere die Aussage von Bill Gates.
\item Auf welche Schwachstellen der gängigen Computersysteme spielt General
Motors in seiner Presseerklärung an? Was wird mit den 13 Aussagen kritisiert?
Welche Kritikpunkte sind berechtigt?
\end{compactenum}