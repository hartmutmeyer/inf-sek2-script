\chapter{Softwaretechnik}
\renewcommand{\chaptertitle}{Softwaretechnik}

\lehead[]{\sf\hspace*{-2.00cm}\textcolor{white}{\colorbox{lightblue}{\makebox[1.60cm][r]{\thechapter}}}\hspace{0.17cm}\textcolor{lightblue}{\chaptertitle}}
\rohead[]{\textcolor{lightblue}{\chaptertitle}\sf\hspace*{0.17cm}\textcolor{white}{\colorbox{lightblue}{\makebox[1.60cm][l]{\thechapter}}}\hspace{-2.00cm}}
%\chead[]{}
\rehead[]{\textcolor{lightblue}{AvHG, Inf, My}}
\lohead[]{\textcolor{lightblue}{AvHG, Inf, My}}

\lstset{style=myJava}

{\em Softwaretechnik} (englischer Fachbegriff: {\em Software Engineering})
beschreibt den Prozess der Software-Erstellung. Das beinhaltet neben der
Programmierung, der Dokumentation  und den Tests auch die Planungs- und
Wartungsphase.

Ohne ein ausgewachsenes Projektmanagement ist dies nicht erfolgreich zu leisten.

\section{Die Entwicklung eines Video-Überwachungssystems oder\\
         warum die Firma LogoSoft Pleite ging}

\subsection{Was vorher geschah}
 
Anmerkung: Alle Namen und Daten in der folgenden Geschichte sind frei erfunden.
Die Geschichte selbst hat sich so ähnlich aber wirklich zugetragen! Die Autorin
hat nach ihren Erlebnissen beschlossen den Job als Softwareentwicklerin
aufzugeben, um in der Schule für die Ausbildung fähigerer Generationen zu sorgen.

September 2000: Die Firma LogoSoft ist ein kleines Software-Unternehmen, das
seit ca.\ 10 Jahren vergeblich versucht, Geld durch den Verkauf von
Videokonferenzsystemen zu verdienen. Die Videokonferenzsysteme wurden in der
Firma ohne festen Auftrag entwickelt, in der Hoffnung, dass sie bei anderen
Unternehmen reißenden Absatz finden. Die hergestellte Software ist qualitativ
gut, aber der jährliche Umsatz von ca.\ 30 Systemen zu je 1500 € reicht nicht
einmal aus, um die Gehälter der ca.\ 40 Mitarbeiter in einem einzigen Monat zu
bezahlen. Glücklicherweise wird die Firma von einer Muttergesellschaft aus
England finanziert, die seit Jahren auf den großen Durchbruch hofft. Allerdings
dämmert es den Geschäftsführern in England inzwischen, dass man mit
Videokonferenzsystemen wohl doch nicht das große Geld machen kann. Deshalb hat
der Geschäftsführer der Firma LogoSoft die Anweisung erhalten, in Zukunft keine
eigenen Produkte mehr zu entwickeln. Statt dessen sollen nur noch Produkte
hergestellt werden, die von einem Kunden, d.h.\ einer anderen Firma, in Auftrag
gegeben werden. Die Kosten werden in diesem Fall direkt von der auftraggebenden
Firma bezahlt.

Der Vertriebsleiter der Firma LogoSoft hat Kontakt zu der Firma BlueEye, die
glaubt, dass es eine Marktlücke für preiswerte Video-Überwachungssysteme gibt.
Sie wollen einen PC-basierten Videorekorder entwickeln lassen, der in der Lage
ist, die Aufnahme von 32 im Haus verteilten Kameras aufzuzeichnen. Die
aufgenommenen Bilder sollen auf der Festplatte des PCs abgespeichert werden.

\subsection{Der Vertrag wird abgeschlossen}

Oktober 2000: Der Vertriebsleiter, der Geschäftsführer und der
Abteilungsleiter-Technik verhandeln mit der Firma BlueEye über die Erstellung
des gewünschten Video-Überwachungssystems. Die Anforderungen an das System
werden schriftlich festgelegt. Es wird abgesprochen, dass die 32 Kameras über
eine spezielle Hardware-Karte an den PC angeschlossen werden, die von der
Schwesterfirma HardSoft entwickelt wird. Die Firma LogoSoft übernimmt die
Programmierung der Software auf der Hardware-Karte und der Software, die auf
dem PC läuft. Damit das System so preiswert wie möglich wird, werden billige
Hardware-Komponenten verwendet, die zum aktuellen Zeitpunkt eigentlich schon
veraltet sind. Die Tatsache, dass das System durch die technische
Weiterentwicklung in zwei Jahren keine befriedigende Leistung mehr bringen
wird, wird ignoriert. Hauptsache Kosten sparen! Daher wird auch beschlossen,
dass auf dem PC das Betriebssystem Linux eingesetzt wird, denn dies ist ein
preiswertes Betriebssystem, das im Moment \glqq in\grqq\ ist. Das Betriebssystem
Linux ist allerdings nicht für zeitkritische Anwendungen entwickelt worden. Das
Video-Überwachungssystem muss Monate lang ohne Wartung Bilder aufzeichnen
können. Dabei darf kein einziges Bild verloren gehen. Es wäre untragbar, wenn
gerade das Bild, auf dem vielleicht der Einbrecher zu sehen ist, wegen
Systemfehlern nicht abgespeichert werden würde. Ob das Betriebssystem Linux und
die verwendeten Hardware-Komponenten in der Lage sind, diese Anforderungen zu
erfüllen, wird nicht näher untersucht.

Nachdem die technischen Details geklärt sind, erstellt ein erfahrener
Software-Entwickler, ein sogenannter {\em Senior Developer}, einen Grobplan für
die Entwicklung der Software. Er überlegt, aus welchen Komponenten sich das
System zusammen setzen muss. Dann schätzt er die dafür benötigte Zeit ab. Er
glaubt, das das System von vier gut eingearbeiteten Mitarbeitern in einem
halben Jahr erstellt werden kann. Nach dieser Grob-Kalkulation errechnet die
Vertriebsabteilung einen Festpreis, zu dem die Software an die Firma BlueEye
verkauft wird. Der Preis, die Anforderungen an das System und der Termin, bis
zu dem die Software fertig sein muss, werden in einem Vertrag schriftlich
festgehalten. Die Firma LogoSoft verspricht, die Software bis zum Mai 2001 zu
liefern. Im Sommer 2001 will die Firma BlueEye bereits die ersten
Video-Überwachungssysteme verkaufen.

\subsection{Das Projekt beginnt}

November 2000: Die Arbeit an dem Video-Überwachungssystem in der Firma LogoSoft
beginnt. Da die meisten Mitarbeiter in laufenden Projekten eingebunden sind
(die alle nur wenig Geld einbringen), hat der Abteilungsleiter-Technik
Probleme, die vierköpfige Software-Entwicklungs-Mannschaft zusammen zu stellen.
Er ernennt Kai Menzel zum Projektleiter. Kai Menzel ist ein langjähriger,
erfahrener Mitarbeiter, der jedoch noch nie ein eigenes Projekt geleitet hat.
Als einzigen Vollzeit-Mitarbeiter erhält Kai einen neuen Kollegen der frisch
von der Universität kommt, und noch niemals ein Programm entwickelt hat, das
länger als fünf Din A4 Seiten ist. Die weiteren Mitarbeiter des neuen Projektes
sind zwei Studenten (mit noch geringerer Programmiererfahrung), die halbtags in
der Firma jobben. Alle drei verdienen noch nicht viel und sind für die Firma
LogoSoft preiswerte Arbeitskräfte. Keiner der vier Personen hat jemals zuvor
Software auf dem Betriebssystem Linux erstellt.

Da man bei großen Projekten nicht einfach so drauf los programmieren kann,
erstellt Kai Menzel als erstes eine gründliche Projektplanung. Er vertraut
dabei voll auf die zuvor von dem Senior Developer erstellte Grob-Planung. Er
definiert genau, welche Aufgaben die einzelnen Software-Komponenten erfüllen
müssen. Die Vorgaben werden grafisch und verbal festgehalten. Diese Planung
dauert mehrere Wochen. Anschließend entscheidet Kai, welcher Mitarbeiter welche
Komponenten erstellt und stellt dafür einen genauen Zeitplan auf.

\subsection{Ablauf des ersten halben Jahres}

Dezember 2000: Die drei Mitarbeiter von Kai Menzel haben ihre Teilaufgaben
zugewiesen bekommen und können mit ihrer Arbeit beginnen. Jeder der Mitarbeiter
studiert genau die von Kai festgesetzten Anforderungen und macht sich dann
einen detaillierten Plan, wie er die ihm zugewiesene Komponente programmieren
will. Nach zwei bis drei Wochen Einarbeitung und Planung sind alle Mitarbeiter
so weit, dass sie mit der Programmierung loslegen können. Wenn jetzt noch die
benötigte Hardware da wäre \ldots

Februar 2001: Kai Menzel und seine Mitarbeiter haben große Probleme mit den von
der Firma HardSoft gelieferten Prototypen für die Hardware-Karte. Es wird
mehrfach ein neuer Prototyp geliefert und immer wieder stellen sie fest, dass
die Karte fehlerhaft ist. Bei den Hardware-Tests geht eine Menge Zeit verloren.
Außerdem stellt sich heraus, dass die Programmierung durch die billigen
Hardware-Komponenten zeitaufwendiger ist als angenommen. Der Zeitplan ist nicht
mehr einzuhalten.

April 2001: Das Projekt hinkt dem ursprünglichen Zeitplan hoffnungslos
hinterher. Um noch zu retten was zu retten ist, beschließt der
Abteilungsleiter-Technik, zwei erfahrene Softwareentwickler von ihren laufenden
Projekten abzuziehen und sie in das Projekt \glqq Video-Überwachungssystem\grqq\
zu stecken. Die beiden neuen Projekt-Mitarbeiter, Marco Zwickel und Eva
Kaufmann, beginnen sich einzuarbeiten.


\subsection{Das Projekt muss verlängert werden}

Mai 2001: Die beiden neuen Projekt-Mitarbeiter, Marco und Eva, sind jetzt im
Bilde und sind vom Zustand des Projektes entsetzt. Marco soll mit dem
Frischling von der Uni zusammen die Software für die Hardware-Karte fertig
stellen. Er stellt fest, das der Neuling aus Unerfahrenheit viele Fehler
eingebaut hat. Ein großer Teil seiner Komponenten muss noch einmal überarbeitet
werden. Eva hat die Aufgabe, die PC-Software zu schreiben, um die sich bisher
überhaupt noch niemand gekümmert hat. Sie stellt fest, dass die veranschlagte
Zeitschätzung für die Komponente unrealistisch niedrig ist. Statt wie
vorgegeben einen Monat wird sie mindestens drei Monate zur Erstellung der
Software benötigen. Außerdem stellen Marco und Eva fest, dass bei der Planung
viele wichtige Überlegungen unterlassen wurden. Beide haben große Bedenken, ob
das System, so wie es geplant ist, stabil laufen wird.

Juni 2001: Auf dem Betriebssystem Linux treten unerwartete Probleme auf. Hätte
man das System vielleicht doch vorher auf Tauglichkeit untersuchen sollen? Ein
einzelner Mitarbeiter wird ca. drei Monate brauchen, um diese Probleme zu
beseitigen. Der Abteilungsleiter-Technik sieht sich gezwungen, einen weiteren
Softwareentwickler für das Projekt „Video-Überwachung“ abzustellen. Fast die
gesamte Softwareentwicklungsabteilung ist jetzt in das Projekt eingebunden.


\subsection{Der Geldhahn wird zugedreht}

Juli 2001: Die Muttergesellschaft in England hat keine Geduld mehr mit der
Firma LogoSoft. Einer der Gründe dafür ist sicher auch der katastrophalen
Ablauf des Projekts Video-Überwachung. Es wird beschlossen, dass 15 der 40
Mitarbeitern entlassen werden müssen. Da die Firma auf keinen einzigen
Software-Entwickler verzichten kann, werden in der Abteilung \glqq Technik\grqq\
fünf \glqq einfache\grqq\ Mitarbeiter entlassen, die kein Universitätsdiplom
besitzen. Die entlassenen Techniker wurden bisher für die Wartung der Geräte,
die Kundenbetreuung und den Test von fertigen Software-Komponenten eingesetzt.
Für dieses Aufgaben steht jetzt niemand mehr zur Verfügung. Die Mitarbeiter
sind von der Firmenpolitik irritiert.


\subsection{Technische Mängel}

August 2001: Das geplante Software-System ist fertig gestellt. Aber es
funktioniert nicht. Es stellt sich heraus, dass das System zu langsam ist. Es
ist nicht in der Lage die Bilder der 32 Kameras in der erforderlichen
Geschwindigkeit aufzuzeichnen. In großen Software-Firmen ist es eine
Selbstverständlichkeit, dass zu Beginn der Entwicklung mit einem abgespeckten
Prototyp ausgetestet wird, ob alle Vorstellungen technisch umsetzbar sind. Die
Firma LogoSoft hat jedoch um Kosten zu sparen auf diesen Schritt verzichtet.

Die Projektleiter und der Senior Developer untersuchen hektisch, wie das System
so abgeändert werden kann, dass die vorgegebenen Anforderungen doch noch
erfüllt werden können. Schließlich finden sie eine Lösung. Einige Komponenten
auf der Hardware-Karte müssen verändert werden, und die Software muss
entsprechend umgeschrieben werden. Die Erweiterung wird mehrere Monate Zeit
kosten.

September 2001: Die in der Firma verbliebenen Mitarbeiter stellen fest, dass
die Gehälter mit Verspätung ausgezahlt werden.


\subsection{Die Test-Phase}

November 2001: Die Software des Video-Überwachungssystems ist fertig gestellt.
Nun beginnt eine mehrere Monate lange Testphase. Nachdem die \glqq
einfachen\grqq\ Techniker von der Firma entlassen wurden, müssen die Tests von
den Software-Entwicklern selbst durchgeführt werden. Da die Software-Entwickler
ein höheres Gehalt kriegen, wird die Testphase für die Firma ungewöhnlich teuer.

Bei den Langzeit-Tests, die jetzt durchgeführt werden, kommen naturgemäß noch
eine Reihe Software-Fehler zutage, die von den Software-Entwicklern behoben
werden müssen. Nach einer Fehlerbehebung müssen immer alle Tests noch einmal
durchgeführt werden, und häufig kommen dann weitere Fehler zutage. Es gibt
mehrere Zyklen mit einer Testphase und einer anschließenden Fehlerbehebung bis
die Software endlich \glqq rund\grqq\ ist.

Während dessen gehen jede Woche neue Kündigungen von Mitarbeitern ein, die sich
wegen der schlechten finanziellen Lage der Firma nach einem neuen Job umgesehen
haben. Auch aus der Führungsetage kündigen viele Mitarbeiter.


\subsection{Das Ende}

März 2002: Das Video-Überwachungssystem ist endlich fertig gestellt.

Die Muttergesellschaft in England beschließt, die Firma LogoSoft nicht mehr
finanziell zu unterstützen. Alle Mitarbeiter werden zum nächstmöglichen
Zeitpunkt entlassen.

Ob die Firma BlueEye es schafft, ihr Video-Überwachungssystem erfolgreich zu
verkaufen oder ob sie ebenfalls Pleite geht, bleibt noch abzuwarten.
