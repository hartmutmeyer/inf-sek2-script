\clearpage

\lehead[]{\sf\hspace*{-2.00cm}\textcolor{white}{\colorbox{lightblue}{\makebox[1.60cm][r]{\thechapter}}}\hspace{0.17cm}\textcolor{lightblue}{\chaptertitle}}
\rohead[]{\textcolor{lightblue}{\chaptertitle}\sf\hspace*{0.17cm}\textcolor{white}{\colorbox{lightblue}{\makebox[1.60cm][l]{\thechapter}}}\hspace{-2.00cm}}
%\chead[]{}
\rehead[]{\textcolor{lightblue}{AvHG, Inf, My}}
\lohead[]{\textcolor{lightblue}{AvHG, Inf, My}}

\section{Konfigurations-Dateien (\myClass{java.util.Properties})}

Mit der Klasse \myClass{Properties} aus dem Package \myPackage{java.util} kann
man in einer Datei Einträge unter einem Stichwort abspeichern. Zum Beispiel
könnten die Punktzahl und der Namen des Highscore-Inhabers in einer Datei
folgendermaßen gespeichert werden:

\begin{lstlisting}
#Sun Jun 06 18:01:53 CEST 2010
Name=Daniela
Punkte=3015
\end{lstlisting}

Die Klasse \myClass{Properties} übernimmt die Verwaltung der Eigenschaften (im
Beispiel heißen die Eigenschaften \myUserInput{Name} und \myUserInput{Punkte}).

Zum Abspeichern der Einträge in einer Datei benötigt man außerdem ein
Objekt der Klasse \myClass{FileOutputStream}, das die Ausgabedatei
erzeugt\footnote{In der von uns verwendeten Form muss die Datei bereits vorab
existieren -- soll es auch mit noch nicht existierenden Dateien funktionieren,
darf man die Datei nicht mit \lstinline|getClass().getResource()| bestimmen.}
und das eigentliche Schreiben der Daten in die Datei übernimmt. Falls die
Ausgabedatei schon existiert, wird sie überschrieben.

Zum Einlesen der Einträge aus einer Datei, benötigt man zusätzlich ein Objekt
der Klasse \myClass{FileInputStream}, das die Datei zum Lesen öffnet und die
Daten einliest und sie anschließend zur Interpretation an die Klasse
\myClass{Properties} weiterleitet.

\begin{compactenum}[a)]
\item Highscore in Datei speichern

Der folgende Code-Auszug liest den Namen und die Punktzahl des Besten aus zwei
Textfeldern aus und speichert sie in der Datei
\myFile{highscore.properties}\footnote{Die Dateiendung \myFile{*.properties}
ist nicht zwingend, aber üblich. Grundsätzlich darf die Datei beliebig benannt
werden.} ab:

\begin{lstlisting}
public void laden() {
    URL url = getClass().getResource("highscore.properties");
    if (url == null) {
        System.out.println("Fehler beim Lesen: Die Datei existiert nicht!");
        return;
    }
    try (InputStream is = new FileInputStream(url.getFile());
            InputStreamReader in = new InputStreamReader(is, "UTF-8")) {
        Properties properties = new Properties();	
        properties.load(in);
        String name = properties.getProperty("Name");
        String punkte = properties.getProperty("Punkte");
        tfName.setText(name);
        tfPunkte.setText(punkte);
    } catch (IOException e) {
        System.out.println(e.getMessage());
    }
}
\end{lstlisting}

\item Highscore aus Datei einlesen

Der folgende Code-Auszug liest den Namen und die Punktzahl aus der der Datei
\myFile{highscore.properties} aus und schreibt sie in zwei Textfelder:

\begin{lstlisting}
public void speichern() {
    URL url = getClass().getResource("highscore.properties");
    if (url == null) {
        System.out.println("Fehler beim Schreiben: Die Datei existiert nicht!");
        return;
    }
    try (OutputStream os = new FileOutputStream(url.getFile());
            OutputStreamWriter out = new OutputStreamWriter(os, "UTF-8")) {
        Properties properties = new Properties();
        String name = tfName.getText();
        String punkte = tfPunkte.getText();
        properties.setProperty("Name", name);
        properties.setProperty("Punkte", punkte);
        properties.store(out, null);
        out.flush();
    } catch (Exception e) {
        System.out.println(e.getMessage());
    }
}
\end{lstlisting}
\end{compactenum}