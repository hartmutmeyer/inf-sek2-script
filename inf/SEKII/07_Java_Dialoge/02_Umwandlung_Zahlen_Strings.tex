\section{Umwandlung String $\leftrightarrow$ Zahl}

Bei der Verwendung von Dialogen gibt es oft die Notwendigkeit zwischen Strings
und Zahlenwerten zu konvertieren.

\subsection{Umwandlung von Strings in Zahlen}

\subsubsection{String $\rightarrow$ int}

Die Klasse \myClass{Integer} besitzt folgende Methode zur Umwandlung eines
Strings in eine ganze Zahl:

\begin{lstlisting}
public static int parseInt(String s)
\end{lstlisting}

Beispiel:

\begin{lstlisting}
String text = "10";
int zahl = Integer.parseInt(text);
\end{lstlisting}


\subsubsection{String $\rightarrow$ double}

Die Klasse \myClass{Double} besitzt folgende Methode zur Umwandlung eines
Strings in eine Fließkommazahl:

\begin{lstlisting}
public static double parseDouble(String s)
\end{lstlisting}

Beispiel:

\begin{lstlisting}
String text = "5.25";
double zahl = Double.parseDouble(text);
\end{lstlisting}


\subsection{Umwandlung von Zahlen in Strings}

Am einfachsten wandelt man eine Zahl in einen String um, in dem man sie mit
\verb|+| an einen Leerstring anhängt.

Beispiel:

\begin{lstlisting}
int zahl = 10;
String text = "" + zahl;
\end{lstlisting}

Dabei ist es unerheblich, ob es sich um einen Ganzzahl- oder einen
Fließkommazahlwert handelt.
