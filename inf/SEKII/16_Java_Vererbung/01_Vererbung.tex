\chapter{Vererbung}
\renewcommand{\chaptertitle}{Vererbung}

\lehead[]{\sf\hspace*{-2.00cm}\textcolor{white}{\colorbox{lightblue}{\makebox[1.60cm][r]{\thechapter}}}\hspace{0.17cm}\textcolor{lightblue}{\chaptertitle}}
\rohead[]{\textcolor{lightblue}{\chaptertitle}\sf\hspace*{0.17cm}\textcolor{white}{\colorbox{lightblue}{\makebox[1.60cm][l]{\thechapter}}}\hspace{-2.00cm}}
%\chead[]{}
\rehead[]{\textcolor{lightblue}{AvHG, Inf, My}}
\lohead[]{\textcolor{lightblue}{AvHG, Inf, My}}

\lstset{style=myJava}

\section{Wie man eine Klasse ableitet}

Bei der Klassendeklaration kann man mit dem Schlüsselwort \verb|extends| eine
Oberklasse angeben, von der die neue Klasse alle Variablen und Methoden erbt:

\begin{minipage}{0.65\textwidth}
\begin{lstlisting}
[public] class KlassenName [extends Super-Klasse] {æ
    // ...
æ}
\end{lstlisting}
\end{minipage}
\hfill
\begin{minipage}{0.25\textwidth}
\lstinline|[...]| bedeutet, dass die Angaben zwischen den eckigen
Klammern optional sind.
\end{minipage}

Wenn man keine Oberklasse angibt, dann wird die Klasse automatisch von der
Klasse \myClass{Object} aus dem Package \myPackage{java.lang} abgeleitet. Diese
Klasse stellt einige Java-Standardfunktionalitäten bereit.


\section{Überschreiben von Methoden}

Die neue Klasse erbt automatisch alle Methoden der Oberklasse und kann neue
Methoden dazu definieren. Sie kann jedoch auch die Arbeitsweise von Methoden der
Oberklasse verändern, in dem sie sie überschreibt. Eine Methode wird
überschrieben, in dem in der abgeleiteten Klasse exakt dieselbe Methode noch
einmal deklariert wird. Name, Parameter und Rückgabewert müssen dabei
übereinstimmen.

\emph{Werden nun beide Methoden ausgeführt (die Methode in der Oberklasse und
die Methode in der abgeleiteten Klasse)?}

Nein. Es wird nur die Methode in der abgeleiteten Klasse ausgeführt. Wenn ein
Programmierer eine Methode auf ein Objekt anwendet, dann sucht das Java-System
zunächst in der Klasse des Objektes nach einer entsprechenden Methode. Wenn es
diese findet, wird die Methode ausgeführt und die Aufgabe ist erledigt. Wenn
die Methode jedoch in der eigentlichen Klasse nicht vorhanden ist, dann sucht
das System in der Oberklasse nach einer entsprechenden Methode und führt diese
aus. Ist auch dort keine Methode definiert, wird die Ober-Oberklasse
durchsucht, usw.

\emph{Was ist aber, wenn man die Methode der Oberklasse nur erweitern möchte?
Muss man den ganzen Code neu schreiben, oder kann man die Methode der
Oberklasse irgendwie aufrufen?} 

Wenn man die Methode der Oberklasse nicht völlig ändern sondern nur erweitern
möchte, sollte man auf jeden Fall die Methode aus der Oberklasse aufrufen. Wir
haben ja gelernt, dass derselbe Code nicht an mehreren Stellen im Programm
stehen sollte. Mit folgendem Befehl kann man explizit eine Methode aus der
Oberklasse aufrufen:

\begin{lstlisting}
super.methode(parameter);
\end{lstlisting}


\subsection{Noch ein paar Besonderheiten}

\begin{compactitem}
\item Methoden mit dem Schlüsselwort \verb|static| können nicht überschrieben
werden, da sie sich ja auf die gesamte Klasse und nicht auf einzelne Objekte
beziehen.
\item Mann kann eine Methode in der Oberklasse mit \verb|final| markieren, um zu
verbieten, dass sie in abgeleiteten Klassen überschrieben wird.
\end{compactitem}


\subsection{Konstruktor}

Mit folgendem Befehl kann man innerhalb eines Konstruktors den Konstruktor der
Oberklasse aufrufen:

\begin{lstlisting}
super(parameter);
\end{lstlisting}

Konstruktoren werden nicht wie andere Methoden vererbt. Wenn man in der
Oberklasse einen Konstruktor mit Parametern definiert hat, so ist dieser nicht
automatisch in der abgeleiteten Klasse vorhanden.

Die Regeln für Konstruktoren lauten folgendermaßen:

\begin{compactenum}[1.]
\item Wenn in einer Klasse kein Konstruktor definiert ist, dann erzeugt das
System automatisch einen Standardkonstruktor ohne Parameter, der folgendermaßen
aussieht:

\begin{lstlisting}
Klasse() {
    super();
}
\end{lstlisting}

\item Falls man einen oder mehrere Konstruktoren selbst geschrieben hat, dann
existiert der parameterlose Konstruktor (Standardkonstruktor) nicht, wenn man
ihn nicht explizit geschrieben hat.
\item Wenn man einen eigenen Konstruktor schreibt, dann muss man in der ersten
Zeile des Konstruktors den gewünschten Konstruktor der Oberklasse aufrufen.
Wenn man dies nicht tut, dann wird automatisch der parameterlose Konstruktor
der Oberklasse aufgerufen.

\emph{Achtung:} Falls in der Oberklasse gar kein parameterloser Konstruktor
definiert ist, erhält man einen Compiler-Fehler!
\end{compactenum}


\section{Datenkapselung: vollständiger Überblick}

Bisher kennen wir für die Datenkapselung nur die Schlüsselworte \verb|public|
und \verb|private|. Mit \verb|private| gekennzeichnete Methoden können jedoch in
einer abgeleiteten Klasse nicht benutzt werden. Wie kann man dafür sorgen, dass
eine Methode zwar von abgeleiteten Klassen benutzt werden kann, aber für andere
Klassen (zumindest aus anderen Packages) versteckt ist? Antwort: man benutzt
das Schlüsselwort \verb|protected|.

Die folgende Tabelle gibt einen vollständigen Überblick über die Möglichkeiten
der Datenkapselung in Java:

\begin{center}
\bgroup
\def\arraystretch{1.2}
\begin{tabular}{|l|c|c|c|c|}
\hline
\textbf{Attribut (UML-Zeichen)} & \textbf{Klasse} & \textbf{abgeleitete
Klasse} & \textbf{Package} & \textbf{andere Packages} \\ \hline
\verb|private| (--)   &  +  & --    & -- & -- \\ \hline
\verb|protected| (\#) &  +  & + (*) &  + & -- \\ \hline
\verb|public| (+)     &  +  &  +    &  + & +  \\ \hline
<kein Angabe>         &  +  & --    &  + & -- \\ \hline
\end{tabular}
\egroup
\end{center}

(*) Ableitung in fremdem Paket: Zugriff nur bei Objekten vom Typ der
abgeleiteten Klasse nicht in Objekten vom Typ der Basisklasse.
