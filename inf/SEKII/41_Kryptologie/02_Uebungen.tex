\clearpage

\lehead[]{\sf\hspace*{-2.00cm}\textcolor{white}{\colorbox{lightblue}{\parbox[c][0.70cm][b]{1.60cm}{
\makebox[1.60cm][r]{\thechapter}\\ \makebox[1.60cm][r]{ÜBUNG}}}}\hspace{0.17cm}\textcolor{lightblue}{\chaptertitle}}
\rohead[]{\textcolor{lightblue}{\chaptertitle}\sf\hspace*{0.17cm}\textcolor{white}{\colorbox{lightblue}{\parbox[c][0.70cm][b]{1.60cm}{\thechapter\\
ÜBUNG}}}\hspace{-2.00cm}}
%\chead[]{}
\rehead[]{\textcolor{lightblue}{AvHG, Inf, My}}
\lohead[]{\textcolor{lightblue}{AvHG, Inf, My}}

\section{Kryptologie -- Übungen}

\subsection{Aufgabe 1: Caesar-Verfahren I}

Wie viele Schlüssel-Möglichkeiten gibt es beim Caesar-Verfahren?


\subsection{Aufgabe 2: Caesar-Verfahren II}

Verschlüssele von Hand ein Wort mit dem Caesar-Verfahren. Das Wort darf keine
Umlaute enthalten. Tausche anschließend das Wort und den benutzten Schlüssel
(das heißt die Anzahl der Buchstaben, um die das Alphabet verschoben wurde) mit
einem deiner Sitznachbarn aus und versuche das Wort deines Nachbarn zu dekodieren.


\subsection{Aufgabe 3: Caesar-Verfahren III}

Der folgende Text wurde mit dem Caesar-Verfahren verschlüsselt. Versuche
herauszufinden um wie viele Buchstaben das Alphabet verschoben wurde und
dekodiere den Anfang des Textes.

\begin{quotation}
\noindent
Ajwxhmqzjxxjqzslxyjhmsnp nxy jns xjmw xufssjsijx Ymjrf, ifx snhmy jwxy xjny
Gjlnss ijx Htruzyjwejnyfqyjwx wjqjafsy nxy. Xhmts Ozqnzx Hfjxfwx mfy fs xjnsj
Ajwgzjsijyjs ljmjnrj Gtyxhmfkyjs ljxhmnhpy. Inj Yjcyj bzwij jnsjr Gtyjs ns
ajwxhmqzjxxjqyjw Ktwr fsajwywfzy. Kfqqx ijw Gtyj ats Kjnsijs zjgjwkfqqjs bzwij,
ptssyjs xnj fzx ijs fgljkfsljsjs Sfhmwnhmyjs snhmy xhmqfz bjwijs. Ozqnzx Hfjxfw
bfw jgjs enjrqnhm ljsnfq.
\end{quotation}


\subsection{Aufgabe 4: Substitutionsverfahren I}

\begin{compactenum}[a)]
\item Nenne für das Substitutionsverfahren einen Schlüssel, bei dem es keine
echte Verschlüsselung gibt und alle Wörter unverändert bleiben.
\item Wie viele verschiedene Verschlüsselungsmöglichkeiten gibt es mit dem
 Substitutionsverfahren?
\end{compactenum}


\subsection{Aufgabe 5: Substitutionsverfahren II}

Verschlüssele von Hand ein Wort mit dem Substitutionsverfahren. Das Wort darf
keine Umlaute enthalten. Tausche anschließend das Wort und den benutzen
Schlüssel mit einem deiner Sitznachbarn aus und versuche das Wort deines
Nachbarn zu dekodieren.


\subsection{Aufgabe 6: Substitutionsverfahren III}

Wie könnte man das Substitutionsverfahren entschlüsseln?


\subsection{Aufgabe 7: Substitutionsverfahren IV}

Im Kursverzeichnis findest du das Tool \myUserInput{textanalyse.exe}. Kopiere
zunächst einige unverschlüsselte Texte in das Eingabefeld und untersuche die
Häufigkeitsverteilung der Buchstaben. Am schnellsten geht dies, wenn du
irgendwelche Texte markierst (zum Beispiel im Internet-Browser) und in das
Textfeld einfügst. Untersuche anschließend den nach der Substitutionsmethode
kodierten Text in der Datei \myUserInput{Substitution.txt}. Versuche die
Geheimtext-Buchstaben für diejenigen Klartext-Buchstaben herauszubekommen, die
besonders häufig oder besonders selten auftreten.


\subsection{Aufgabe 8: Vigenère-Verfahren I}

Verschlüssele von Hand ein Wort mit dem Vigenère-Verfahren. Das Wort darf keine
Umlaute enthalten. Tausche anschließend das Wort und den benutzen Schlüssel mit
einem deiner Sitznachbarn aus und versuche das Wort deines Nachbarn zu
dekodieren.


\subsection{Aufgabe 9: Vigenère-Verfahren II}

Im Kursverzeichnis findest du den Text \myUserInput{Vigenere.txt}. Untersuche
mit Hilfe des Textanalyse-Tools, ob auch bei einem mit dem Vigenère-Verfahren
verschlüsselten Text Buchstaben durch ihre Häufigkeitsverteilung hervortreten.
Überlege dir anschließend, wie man einen mit Vigenère verschlüsselten Text
dekodieren kann.


\subsection{Aufgabe 10: Programmierübung}

Programmiere eines der drei klassischen Verschlüsselungsverfahren, die wir
kennen gelernt haben (Caesar, Substitution oder Vigenère). Schreibe dazu ein
Programm, dass einen Text aus einer Datei einliest und ihn verschlüsselt in
einer anderen Datei abspeichert. Programmiere anschließend die Dekodierung der
Nachricht, in dem du den verschlüsselten Code aus einer Datei ausliest und ihn
entschlüsselt in eine zweite Datei schreibst.


\subsection{Aufgabe 11: Asymmetrische Verfahren}

Stelle die asymmetrische Kodierung einer Nachricht in einem Schaubild dar. Das
Schaubild soll einen Kreislauf zeigen, der bei der ursprünglichen Nachricht
beginnt und dort auch wieder endet.


\subsection{Aufgabe 22: Digitale Signatur}

Erkläre was eine digitale Unterschrift ist. Nach welchem Prinzip werden
digitale Unterschriften erstellt?


\subsection{Aufgabe 13: RSA}

Das Wort \myUserInput{HALLO} soll nach dem RSA Verfahren verschlüsselt und
wieder entschlüsselt werden.

\begin{compactenum}[a)]
\item Wandle die Buchstaben in ihren ASCII-Code um. 
\item Berechne den Public Key und den Private Key für die Kodierung. Als
Primzahlen werden die Zahlen $p=13$ und $q=19$ gewählt. Setze $e$ auf 5 und $d$
auf 173. Überprüfe, ob $e$ und $d$ die notwendigen Bedingungen erfüllen.
\item Verschlüssele jeden Buchstaben einzeln mit dem unter b) berechneten Public
 Key.
\item Entschlüssele die unter c) erzeugten Code-Zahlen mit dem unter b)
berechneten Private Key. Die dabei zu berechnenden Zahlen werden so hoch, dass
sie weder von einem normalen Taschenrechner noch von den in der
Programmiersprache Java verfügbaren Datentypen gespeichert werden können.
Benutze deshalb zur Berechnung den CAS-Modus (CAS steht
für Computer-Algebra-System) von \myUserInput{GeoGebra}. Diesen erreichst du in
GeoGebra über \myUserInput{Ansicht} $\rightarrow$ \myUserInput{CAS}. Die
Rechnungen werden dann in die Eingabezeile eingetippt.

Für $x^y$ schreibt man im CAS-Fenster von GeoGebra \lstinline|x^y|.

Für $x \bmod y$ schreibt man im CAS-Fenster von GeoGebra \lstinline|mod(x,y)|.
\item Ist es möglich, diese Berechnungen auch in einem Java-Programm zu
implementieren, obwohl die Java-Datentypen zu klein sind, um die Zahlen zu
fassen?
\end{compactenum}


\subsection{Aufgabe 14: Vergleich von symmetrischer und asymmetrischer
Verschlüsselung} 
Liste die Vor- und Nachteile der asymmetrischen Verschlüsselung
gegenüber der symmetrischen Verschlüsselung auf.


\subsection{Aufgabe 15: Web of Trust}

Die Schwachstelle der asymmetrischen Verschlüsselung ist, dass man nicht immer
sicher sein kann, ob ein veröffentlichter Public Key wirklich dem angegebenen
Anwender gehört. Informiere dich im Internet auf den deutschen PGP-Seiten
darüber, wie die Vertrauenswürdigkeit eines Schlüssels sichergestellt wird.
