\makeatletter
\setlength{\arrayrulewidth}{1pt}
% Darstellung der Tabelle ohne Datentyp
% 1 = Label
% 2 = Tabellenname
% 3 = Inhalt
\newcommand{\ERMTable}[3]{%
 \rnode{#1}{%
  \begin{tabular}{|l|l|}%
    \hline%
    \multicolumn{2}{|c|}{\textbf{#2}}  \\ \hline%
    #3%
    \hline%
  \end{tabular}%
 }%
}%


% Darstellung der Tabelle mit Datentyp
% 1 = Label
% 2 = Tabellenname
% 3 = Inhalt
\newcommand{\ERMTableDatentyp}[3]{%
 \rnode{#1}{
  \begin{tabular}{|l|ll|}%
   \hline%
   \multicolumn{3}{|c|}{\textbf{#2}}  \\ \hline%
   #3%
   \hline
  \end{tabular}%
 }%
}%

\newcommand{\ermiconN}{%
 \psline(-0.2,-0.3)(0.2,-0.3)			% Strich zur Mitte hin
 \psline(-0.1,0)(0,-0.3)			% Linke Kante
 \psline(0.1,0)(0,-0.3)				% Rechte Kante
}

\newcommand{\ermiconNurN}{%
 \psline(-0.1,0)(0,-0.3)			% Linke Kante
 \psline(0.1,0)(0,-0.3)				% Rechte Kante
}

\newcommand{\ermiconON}{
 \psellipse*[linecolor=white](0,-0.4)(0.15,0.1)	%Hintergrund übermalen!
 \psellipse[linecolor=black](0,-0.4)(0.15,0.1)	%Rand zeichnen
 \psline(-0.1,0)(0,-0.3)			% Linke Kante
 \psline(0.1,0)(0,-0.3)				% Rechte Kante
}


\newcommand{\ermiconEins}{%
 \psline(-0.2,-0.3)(0.2,-0.3)			% Strich zur Mitte hin
 \psline(-0.2,-0.2)(0.2,-0.2)			% Strich auf der Tabellenseite
}

\newcommand{\ermiconNurEins}{%
 \psline(-0.2,-0.2)(0.2,-0.2)			% Strich auf der Tabellenseite
}


\newcommand{\ermiconOEins}{%
 \psline(-0.2,-0.2)(0.2,-0.2)			% Strich auf der Tabellenseite
 \psellipse*[linecolor=white](0,-0.4)(0.15,0.1)	%Hintergrund übermalen!
 \psellipse[linecolor=black](0,-0.4)(0.15,0.1)	%Rand zeichnen
}


\newcommand{\ermputicon}[2][]{
 \psset{npos=0,nrot=90}%
 \psset{#1}%
 \@ifundefined{ermicon#2}{%
  \typeout{^^Fehler : Icon nicht definiert "#2"^^J}
 }%
 {}%
 \ncput{\@nameuse{ermicon#2}}
}


% Nouvelles d\'efinition pour les interconnexions
%%%%%%%%%%%%%%%%%%%%%%%%%%%%%%%%%%%%%%%%%%%%%%%%%%%%%%%%%%%%%%%%%%%%%%

% Ces connecteurs sont une surcouche aux diff\'erents connecteurs
% propos\'es par pstricks.
% Leur but est de simplifier le trac\'e des liens en se restreingnant aux
% directions horizontales et verticales.
%
% Le principe est d'indiquer dans le nom même de la commande le nombre
% de segments à tracer et leur direction :
%  - E, W, N, S pour Est, West, North, Sud
%  - H, V pour Horizontal; Vertical
%  - D pour diagonal
%  - X pour indiff\'erent

% Un seul segments
% Effet de bord : en fait, un deuxieme segment est dessin\'e. Mais celui
% n'apparait g\'en\'eralement pas car il longe la frontière de boite
\newpsobject{ncE}{ncangle}{angleA=0,angleB=180,armB=0,npos=0.5,nodesepB=-0.5pt}
% \newpsobject{ncE}{ncangle}{angleA=0,angleB=180,armB=0,npos=0.5}
\newpsobject{ncW}{ncangle}{angleA=180,angleB=0,armB=0,npos=0.5}
\newpsobject{ncN}{ncangle}{angleA=90,angleB=-90,armB=0,npos=0.5}
\newpsobject{ncS}{ncangle}{angleA=-90,angleB=90,armB=0,npos=0.5}

% Deux segments
\newpsobject{ncEN}{ncangle}{angleA=0,angleB=-90,armB=0}
\newpsobject{ncES}{ncangle}{angleA=0,angleB=90,armB=0}
\newpsobject{ncWN}{ncangle}{angleA=180,angleB=-90,armB=0}
\newpsobject{ncWS}{ncangle}{angleA=180,angleB=90,armB=0}
\newpsobject{ncNE}{ncangle}{angleA=90,angleB=180,armB=0}
\newpsobject{ncNW}{ncangle}{angleA=90,angleB=0,armB=0}
\newpsobject{ncSE}{ncangle}{angleA=-90,angleB=180,armB=0}
\newpsobject{ncSW}{ncangle}{angleA=-90,angleB=0,armB=0}

% Trois segments
% On peut utiliser armA ou armB pour imposer la longueur des extr\'emit\'es
%
% remplace \ncbar (connecteurs en forme de U)
\newpsobject{ncEVW}{ncangles}{angleA=0,angleB=0}
\newpsobject{ncWVE}{ncangles}{angleA=180,angleB=180}
\newpsobject{ncSHN}{ncangles}{angleA=-90,angleB=-90}
\newpsobject{ncNHS}{ncangles}{angleA=90,angleB=90}

% connecteurs en forme de Z (mais \`a angles droits)
\newpsobject{ncEVE}{ncangles}{angleA=0,angleB=180}
\newpsobject{ncWVW}{ncangles}{angleA=180,angleB=0}
\newpsobject{ncNHN}{ncangles}{angleA=90,angleB=-90}
\newpsobject{ncSHS}{ncangles}{angleA=-90,angleB=90}


% connecteurs 3 segments dont segment median en diagonale (incomplet)
\newpsobject{ncEDE}{ncdiag}{angleA=0,angleB=180}
\newpsobject{ncWDW}{ncdiag}{angleA=180,angleB=0}
\newpsobject{ncNDN}{ncdiag}{angleA=90,angleB=-90}
\newpsobject{ncSDS}{ncdiag}{angleA=-90,angleB=90}


% quatre (voire trois) segments :
\newpsobject{ncSXE}{ncangles}{angleA=-90,angleB=180}
\newpsobject{ncSXW}{ncangles}{angleA=-90,angleB=0}
\newpsobject{ncEXS}{ncangles}{angleA=0,angleB=90}
\newpsobject{ncEXN}{ncangles}{angleA=0,angleB=-90}
\newpsobject{ncWXS}{ncangles}{angleA=180,angleB=90}
\newpsobject{ncWXN}{ncangles}{angleA=180,angleB=-90}
\newpsobject{ncNXE}{ncangles}{angleA=90,angleB=180}
\newpsobject{ncNXW}{ncangles}{angleA=90,angleB=0}

\makeatother
