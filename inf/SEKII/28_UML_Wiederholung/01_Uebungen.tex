\chapter{UML Wiederholung}
\renewcommand{\chaptertitle}{UML Wiederholung}


\lehead[]{\sf\hspace*{-2.00cm}\textcolor{white}{\colorbox{lightblue}{\parbox[c][0.70cm][b]{1.60cm}{
\makebox[1.60cm][r]{\thechapter}\\ \makebox[1.60cm][r]{ÜBUNG}}}}\hspace{0.17cm}\textcolor{lightblue}{\chaptertitle}}
\rohead[]{\textcolor{lightblue}{\chaptertitle}\sf\hspace*{0.17cm}\textcolor{white}{\colorbox{lightblue}{\parbox[c][0.70cm][b]{1.60cm}{\thechapter\\
ÜBUNG}}}\hspace{-2.00cm}}
%\chead[]{}
\rehead[]{\textcolor{lightblue}{AvHG, Inf, My}}
\lohead[]{\textcolor{lightblue}{AvHG, Inf, My}}

\section{UML Zustandsdiagramme}

\subsection{Aufgabe 1: Seehund}

In einem Computerspiel soll ein Seehund dargestellt werden. Zeichne dafür ein
UML-Zustandsdiagramm, das die verschiedenen Zustände veranschaulicht, in denen
sich der Seehund des Computerprogramms befinden kann:

\begin{quotation}
\noindent
Zu Beginn schwimmt der Seehund ziellos im Wasser herum. Sobald ein Fisch
erscheint, verfolgt er diesen. Nachdem er den Fisch gefangen hat, isst er ihn
(längerer Vorgang). Falls dies der dritte Fisch war, den er gefangen hat, ist
er satt und ruht sich auf einer Sandbank aus. Die Seehund-Simulation ist damit
zu Ende. Andernfalls schwimmt er weiter, um erneut einen Fisch zu fangen.
\end{quotation}

\subsection{Aufgabe 2: Sprechende Puppe}

Ein Spielzeughersteller entwickelt eine neue sprechende Puppe, die elektronisch
gesteuert werden soll. Du hast die Aufgabe, das Verhalten der Puppe mit einem
UML-Zustandsdiagramm zu beschreiben:

\begin{quotation}
\noindent
Wenn die Puppe hingelegt wird, sagt sie \glqq Gute Nacht\grqq . Wenn sie
aufgerichtet wird, gibt sie den Satz \glqq Spielst du was Schönes mit mir?\grqq\
von sich. Solange sie aufgerichtet ist, kann man der Puppe weitere Sätze
entlocken, indem man sie an den Händen fasst. Wenn man an ihre rechte Hand
fasst, sagt sie \glqq Guten Tag\grqq . Wenn man an die linke Hand fasst, sagt
sie \glqq Mama, ich hab dich lieb.\grqq . Außerdem singt die Puppe ein Lied,
wenn man auf ihren Bauchnabel drückt. Dies funktioniert immer, egal ob sie sich
in aufgerichteter oder liegender Position befindet. Das Lied endet nach einer
Zeitspanne von zwei Minuten.
\end{quotation}


\section{UML Klassendiagramme}

\subsection{Aufgabe 1: Zelten}

Beschreibe mit einem UML-Klassendiagramm die Beziehung zwischen den Klassen
\glqq Mensch\grqq , \glqq Zelt\grqq , \glqq Schlafsack\grqq , und
\glqq Zeltstange\grqq . Gib den Beziehungen einen geeigneten Beziehungsnamen und
trage die Multiplizität ein. Attribute und Methoden der Klassen müssen nicht
beschrieben werden.

\subsection{Aufgabe 2: Gitarren}

Beschreibe mit einem UML-Klassendiagramm die Beziehung zwischen den Klassen
\glqq Gitarre\grqq , \glqq Mensch\grqq , \glqq Saite\grqq , \glqq
Westerngitarre\grqq\ und \glqq Klassische Gitarre\grqq . Gib den Beziehungen
einen geeigneten Beziehungsnamen und trage die Multiplizität ein. Attribute und
Methoden der Klassen müssen nicht beschrieben werden.

\subsection{Aufgabe 3: Landleben}

Es soll ein Computerspiel für Kinder entwickelt werden, dass das Landleben mit
mehreren Bauernhöfen simuliert. Schreibe dazu ein UML-Klassendiagramm:

\begin{quotation}
\noindent
Zu jedem Bauernhof gehören ein Haupthaus, in dem zwei bis sechs Menschen wohnen,
und eine oder mehrere Scheunen, in dem die Tiere untergebracht sind. Von jedem
Gebäude (egal ob Haupthaus oder Scheune) müssen die Maße (Breite und Länge)
abgespeichert werden. Die Menschen unterscheiden sich durch die Eigenschaften
Größe, Haarfarbe und Geschlecht.

\noindent
In einer Scheune leben jeweils fünf bis zehn Tiere. Es gibt Kühe, Pferde und
Schweine. Alle Tiere können laufen, stehen oder liegen. Kühe können darüber
hinaus wiederkäuen. Schweine können grunzen. In einer Scheune stehen in der
Regel eine Reihe von Futtertrögen. Von jedem Futtertrog muss man das
Fassungsvermögen kennen.
\end{quotation}
