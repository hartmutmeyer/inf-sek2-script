\clearpage

\lehead[]{\sf\hspace*{-2.00cm}\textcolor{white}{\colorbox{lightblue}{\makebox[1.60cm][r]{\thechapter}}}\hspace{0.17cm}\textcolor{lightblue}{\chaptertitle}}
\rohead[]{\textcolor{lightblue}{\chaptertitle}\sf\hspace*{0.17cm}\textcolor{white}{\colorbox{lightblue}{\makebox[1.60cm][l]{\thechapter}}}\hspace{-2.00cm}}
%\chead[]{}
\rehead[]{\textcolor{lightblue}{AvHG, Inf, My}}
\lohead[]{\textcolor{lightblue}{AvHG, Inf, My}}

\lstset{style=myJava}

\section{Thread-Synchronisation}

\subsection{Beispiel}

Zwei Threads benutzen einen gemeinsamen Zähler. Dadurch, dass das Betriebssystem
die Ausführung des Codes an jeder beliebigen Stelle unterbrechen kann,
entstehen jedoch Fehler. Die ausgegebenen Zahlen entsprechen nicht der
gewünschten Reihenfolge.

\subsubsection{Haupt-Klasse}

\begin{lstlisting}
public class Synchronisation {
    static int zaehler = 0;
    
    public static void main(final String[] args) {
        MeinThread thread_1 = new MeinThread(1);
        MeinThread thread_2 = new MeinThread(2);
        thread_1.start();
        thread_2.start();
    }
}
\end{lstlisting}

\subsubsection{Thread-Klasse}

\begin{lstlisting}
public class MeinThread extends Thread {
    int threadNummer;

    public MeinThread(int threadNummer) {
        this.threadNummer = threadNummer;
    }

    @Override
    public void run() {
        for (int i=0; i<100; i++) {
            Synchronisation.zaehler++;
            int help = Synchronisation.zaehler;
            æ// Simulation einer aufwändigen Berechnung:
æ            for (long x = 0; x<1000000; x++) {
                Math.sin(x);
            }
            System.out.println("Thread" + threadNummer + ": " + help);
        }
    }
}
\end{lstlisting}

\subsection{Lösung des Problems}

Anweisungen, während deren Ausführung eine Unterbrechung (\emph{Task-Wechsel}
durch den \emph{Scheduler} des Betriebssystems) zu unerwünschten Ergebnissen
führen, werden als \emph{kritische Anweisungen} bezeichnet.

Anweisungen bzw.\ Anweisungsfolgen sind dann kritisch, wenn auf eine gemeinsam
genutzte Ressource (etwa eine Datei, ein Gerät oder auch ein Textfeld) nicht nur
lesend, sonderen auch schreibend zugegriffen wird.

Mann kann einen Block mit kritischen Anweisungen vor Unterbrechungen schützen,
indem man die entsprechenden Anweisungen in einen \emph{synchronisierten Block}
einfasst. Dazu benutzt man das Schlüsselwort \lstinline|synchronized|. Über ein
\emph{Monitor-Objekt} (oft auch einfach \emph{Monitor} genannt) wird der
exklusive Zugriff verwaltet.

Als Monitor-Objekt ist grundsätzlich jedes Objekt geeignet, auf das alle
Threads zugreifen können. Ein Thread kann den synchronisierten Block nur
betreten, wenn kein anderer Thread den entsprechenden Monitor bereits belegt.

\pagebreak

Syntax:

\begin{lstlisting}
synchronized (monitorObjekt) {
    æ//Anweisungen, die nicht unterbrochen werden dürfen
æ}
\end{lstlisting}


Wichtig ist, dass alle Threads das gleiche Objekt als Monitor benutzen!

Achtung: Damit alle Threads zum Zug kommen, sollten immer so wenig Befehle wie
möglich in einem \lstinline|synchronized|-Block stehen!
