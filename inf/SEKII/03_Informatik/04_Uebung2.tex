\clearpage

\lehead[]{\sf\hspace*{-2.00cm}\textcolor{white}{\colorbox{lightblue}{\parbox[c][0.70cm][b]{1.60cm}{
\makebox[1.60cm][r]{\thechapter}\\ \makebox[1.60cm][r]{ÜBUNG}}}}\hspace{0.17cm}\textcolor{lightblue}{\chaptertitle}}
\rohead[]{\textcolor{lightblue}{\chaptertitle}\sf\hspace*{0.17cm}\textcolor{white}{\colorbox{lightblue}{\parbox[c][0.70cm][b]{1.60cm}{\thechapter\\
ÜBUNG}}}\hspace{-2.00cm}}
%\chead[]{}
\rehead[]{\textcolor{lightblue}{AvHG, Inf, My}}
\lohead[]{\textcolor{lightblue}{AvHG, Inf, My}}

\section{Erste Übung -- zweiter Teil}

Noch einmal zu unserem Problem aus der ersten Übung: Die Quadratwurzel einer
beliebigen Zahl soll möglichst genau bestimmt werden \emph{ohne}(!) die Wurzelfunktion des
Taschenrechners zu benutzen.

Setzt euch nochmal in Zweier- oder Dreiergruppen zusammen.

Versucht diesmal euren Algorithmus mit Hilfe von Variablen und
Kontrollstrukturen zu beschreiben. Dabei dürft ihr den Algorithmus im Vergleich
zum ersten Übungsteil auch noch verändern.

Ihr habt wieder 15 Minuten Zeit.
