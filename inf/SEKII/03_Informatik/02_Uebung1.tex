\clearpage

\lehead[]{\sf\hspace*{-2.00cm}\textcolor{white}{\colorbox{lightblue}{\parbox[c][0.70cm][b]{1.60cm}{
\makebox[1.60cm][r]{\thechapter}\\ \makebox[1.60cm][r]{ÜBUNG}}}}\hspace{0.17cm}\textcolor{lightblue}{\chaptertitle}}
\rohead[]{\textcolor{lightblue}{\chaptertitle}\sf\hspace*{0.17cm}\textcolor{white}{\colorbox{lightblue}{\parbox[c][0.70cm][b]{1.60cm}{\thechapter\\
ÜBUNG}}}\hspace{-2.00cm}}
%\chead[]{}
\rehead[]{\textcolor{lightblue}{AvHG, Inf, My}}
\lohead[]{\textcolor{lightblue}{AvHG, Inf, My}}

\section{Erste Übung -- erster Teil}

Für die folgende Übung braucht ihr nur Blatt und Stift. Wer will darf zusätzlich
auch einen Taschenrechner benutzen.

Gegeben ist das folgende Problem: Die Quadratwurzel einer beliebigen Zahl soll
möglichst genau bestimmt werden \emph{ohne}(!) die Wurzelfunktion des
Taschenrechners zu benutzen.

Setzt euch in Zweier- oder Dreiergruppen zusammen und überlegt, wie sich das
Problem lösen lässt. Gesucht ist eine Schritt-für-Schritt Anleitung (mit
anderen Worten: ein \emph{Algorithmus}), wie man die Quadratwurzel einer
beliebigen Zahl finden kann. Dabei spielt es zunächst keine Rolle, wie kurz
bzw.\ effizient der Lösungsweg ist. Für diese Aufgabe habt ihr 15 Minuten Zeit.

Schreibt eure Lösung auf, so dass ihr sie anschließend den anderen vorstellen
könnt.
