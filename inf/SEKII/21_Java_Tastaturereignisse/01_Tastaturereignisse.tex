\chapter{Tastaturereignisse}
\renewcommand{\chaptertitle}{Tastaturereignisse}

\lehead[]{\sf\hspace*{-2.00cm}\textcolor{white}{\colorbox{lightblue}{\makebox[1.60cm][r]{\thechapter}}}\hspace{0.17cm}\textcolor{lightblue}{\chaptertitle}}
\rohead[]{\textcolor{lightblue}{\chaptertitle}\sf\hspace*{0.17cm}\textcolor{white}{\colorbox{lightblue}{\makebox[1.60cm][l]{\thechapter}}}\hspace{-2.00cm}}
%\chead[]{}
\rehead[]{\textcolor{lightblue}{AvHG, Inf, My}}
\lohead[]{\textcolor{lightblue}{AvHG, Inf, My}}

\lstset{style=myJava}

\section{Ereignisbehandlung}

Wenn ein Benutzer eine Taste drückt während das Fenster aktiviert ist, dann
findet im Fenster ein sogenanntes \emph{Ereignis} (engl. \emph{event}) statt.

Wenn der Programmierer von Tastaturereignissen unterrichtet werden möchte, muss
er eine Klasse schreiben, die das Interface \myClass{KeyListener} implementiert.
In diesem Interface sind eine Reihe von Methoden definiert, die das System bei
Tastaturereignissen aufruft:

\bgroup
\def\arraystretch{1.2}
\begin{tabular}{|l|l|}
\hline
\textbf{Methoden des Interfaces \myClass{KeyListener}} & 
\textbf{Wir aufgerufen wenn \ldots}
\\ \hline
\lstinline|public void keyPressed(KeyEvent e)| &
\ldots eine Taste gedrückt wurde.
\\ \hline
\lstinline|public void keyReleased(KeyEvent e)| &
\ldots eine Taste losgelassen wurde.
\\ \hline
\lstinline|public void keyTyped(KeyEvent e)| &
\ldots eine Taste gedrückt und wieder losgelassen wurde.
\\ \hline
\end{tabular}
\egroup

Im Normalfall benötigt man nur die Methode \lstinline|keyPressed()|. Die beiden
anderen Methoden müssen jedoch zumindest mit leeren Methodenrümpfen vorhanden
sein.

Damit die Tastaturereignisse an die Klasse mit dem implementierten Interface
weitergeleitet werden, muss man die Klasse beim Anwendungsfenster „anmelden“.
Dazu besitzt die Klasse \myClass{JFrame} folgende Methode:

\begin{lstlisting}
public void addKeyListener(KeyListener l)
\end{lstlisting}

Beim Aufruf einer Methode aus dem Interface KeyListener übergibt das System ein
Objekt der Klasse \myClass{KeyEvent}. Mit einer Methode dieses Objektes kann man
in Erfahrung bringen, welche Taste vom Benutzer betätigt wurde:

\begin{lstlisting}
public char getKeyChar() 
\end{lstlisting}

Die Methode \lstinline|getKeyChar()| gibt den zur Taste gehörenden Buchstaben
zurück. Der Buchstabe ist vom Datentyp \lstinline|char|.

Alternativ bekommt man mit

\begin{lstlisting}
public int getKeyCode() 

\end{lstlisting}

zurückgemeldet, welche Taste gedrückt wurde. Zu jeder Taste gibt es eine
vordefinierte Konstante der Klasse \myClass{KeyEvent} – etwa \lstinline|VK_UP|
für Pfeiltaste nach oben oder \lstinline|VK_N| für die Taste mit dem Buchstaben
N. Im Unterschied zu \lstinline|getKeyChar()| spielt es dabei keine Rolle welche
Modifier-Taste(n) zusätzlich gedrückt wurden (unter Modifier-Tasten versteht man
die Tasten \lstinline|<Strg>|, \lstinline|<Alt>|, \lstinline|<AltGr>| und die
Hochstelltaste zum Umschalten auf Großbuchstaben). Zur Unterscheidung zwischen
großen und kleinen Buchstaben ist \lstinline|getKeyCode()| folglich ungeeignet.

Achtung: \lstinline|getKeyCode()| liefert nur in \lstinline|keyPressed()| ein
sinnvolles Ergebnis, nicht aber in \lstinline|keyTyped()|!

\section{Datentyp \lstinline|char|}

Buchstaben werden in Java in zwei Byte großen Speicherbereichen abgespeichert,
die mit dem Datentyp \lstinline|char| bezeichnet werden. Intern werden die
Buchstaben als Zahlen von 0 bis 65536 kodiert. Die Unicode-Tabelle weist jeder
Zahl einen entsprechenden Buchstaben zu.

Man kann mit Charactern wie mit Zahlenvariablen arbeiten und ihnen z.B. den
Zahlenwert eines Buchstaben zuweisen:

\begin{lstlisting}
char x = 69;         æ// Buchstabe 'E'
\end{lstlisting}

Da man meistens die Buchstabencodes nicht im Kopf hat, ist es auch möglich,
einen Buchstaben direkt anzugeben. Dazu muss man den Buchstaben in Hochkommata
setzen:

\begin{lstlisting}
char x = 'E';        æ// Zahlencode 69
\end{lstlisting}
