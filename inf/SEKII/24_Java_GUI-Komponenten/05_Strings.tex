\clearpage

\lehead[]{\sf\hspace*{-2.00cm}\textcolor{white}{\colorbox{lightblue}{\makebox[1.60cm][r]{\thechapter}}}\hspace{0.17cm}\textcolor{lightblue}{\chaptertitle}}
\rohead[]{\textcolor{lightblue}{\chaptertitle}\sf\hspace*{0.17cm}\textcolor{white}{\colorbox{lightblue}{\makebox[1.60cm][l]{\thechapter}}}\hspace{-2.00cm}}
%\chead[]{}
\rehead[]{\textcolor{lightblue}{AvHG, Inf, My}}
\lohead[]{\textcolor{lightblue}{AvHG, Inf, My}}

\lstset{style=myJava}

\section{Arbeiten mit Zeichenketten (Strings)}

%Im Zusammenhang mit Benutzeroberflächen muss regelmäßig mit Zeichenketten
%gearbeitet werden.

Im Package \myPackage{java.lang} wird die Klasse \myClass{String} zur Arbeit mit
Zeichenketten definiert. Das Package \myPackage{java.lang} wird immer
automatisch eingebunden und braucht deshalb nicht explizit importiert werden.

Mit Objekten der Klasse \myClass{String} haben wir schon oft gearbeitet. Jede in
doppelte Anführungszeichen gesetzte Zeichenkette (zum Beispiel "Hallo") ist ein
Objekt der Klasse \myClass{String}. Im Gegensatz zu anderen Klassen brauchen
Objekte der Klasse \myClass{String} nicht mit \lstinline|new| erzeugt werden.
Wie bei \lstinline|int| und \lstinline|boolean| Variablen kann man einer
\myClass{String} Variablen einfach einen Wert durch ein Gleichheitszeichen
zuweisen:

\begin{lstlisting}
String s1 = "Hallo";
String s2 = "Guten Tag";
\end{lstlisting}

Mit dem Plus-Zeichen können zwei Strings aneinander gehängt werden:

\begin{lstlisting} 
s1 = s1 + s2;           æ// in s1 steht jetzt: "HalloGuten Tag"; s2 bleibt unverändert 
æs1 = s2 + " Otto";      æ// in s1 steht jetzt: "Guten Tag Otto"
æs1 += "!";              æ// in s1 steht jetzt: "Guten Tag Otto!"
\end{lstlisting}

\begin{framed}
Der Nachteil bei der Arbeit mit Strings ist, dass bei jeder Veränderung der
Zeichenkette ein neues Objekt im Speicher erzeugt wird. Wenn beispielsweise
\lstinline|s1| einen neuen Inhalt bekommt, bleibt der alte Inhalt im Speicher
„als Leiche“ liegen bis irgendwann der sogenannte \emph{Garbage Collector}
(„Müll-Aufsammler“) der Java Virtual Machine Zeit hat, diesen Speicher wieder
frei zu geben. Wenn man eine effizientere Speichernutzung haben will, muss man
mit der Klasse \myClass{StringBuilder} arbeiten. Das lohnt sich aber nur, wenn
man ein Programm schreibt, in dem große Mengen von Zeichenketten verwaltet
werden müssen.
\end{framed}

\vfill

\subsection{Methoden der Klasse \myClass{String} (Auswahl)}

%\subsubsection{Text-Manipulation}

\begin{minipage}{1.0\textwidth} % Seitenumbruch verhindern

\bgroup
\def\arraystretch{1.2}
\begin{tabularx}{\textwidth}{|p{85mm}|X|}
\hline
\lstinline|int length()| & 
Länge des Strings
\\ \hline
\lstinline|String toLowerCase()|

\lstinline|String toUpperCase()| & 
Umwandlung in Klein- bzw.\ Großschreibung
\\ \hline
\lstinline|char charAt(int index)| & 
liefert das Zeichen an Position \lstinline|index|
\\ \hline
\lstinline|String replace(char oldChar, char newChar)| & 
ersetzt überall im String Zeichen \lstinline|oldChar| durch \lstinline|newChar|
\\ \hline
\lstinline|String trim()| & 
entfernt alle Leerzeichen am Anfang und Ende
\\ \hline
\lstinline|int indexOf(int ch)| 

\lstinline|int indexOf(String str)| & 
Anfangsposition des Zeichens oder Teil-Strings. Falls das Zeichen nicht im
String vorkommt, wird -1 zurückgegeben.
\\ \hline
\lstinline|String substring(int beginIndex)| 

\lstinline|String substring(int beginIndex, int endIndex)| & 
liefert den Teil-String von Position \lstinline|beginIndex| bis eine Position
vor \lstinline|endIndex| (oder bis zum Ende)
\\ \hline
\lstinline|boolean contains(String str)| & 
\lstinline|true| falls \lstinline|str| im gegebenen String-Objekt enthalten ist.
Sonst \lstinline|false|
\\ \hline
\end{tabularx}
\egroup

\subsubsection{Vergleich von Strings}

\bgroup
\def\arraystretch{1.2}
\begin{tabularx}{\textwidth}{|p{85mm}|X|}
\hline
\lstinline|==| & 
vergleicht die Speicheradressen der Objekte
\\ \hline
\lstinline|boolean equals(Object anObject)| & 
vergleicht Zeichenketten 
(überschriebene Methode der Superklasse \myClass{Object})
\\ \hline
\lstinline|boolean equalsIgnoreCase(String anotherString)| & 
vergleicht Zeichenketten ohne Beachtung von Groß- und Kleinschreibung
\\ \hline
\lstinline|int compareTo(String anotherString)| & 
Lexikografischer Vergleich:

\lstinline|this < anotherString|: negativer Wert

\lstinline|this = anotherString|: 0

\lstinline|this > anotherString|: positiver Wert
\\ \hline
\end{tabularx}
\egroup

\end{minipage} % Seitenumbruch verhindern

\pagebreak

\section{Methoden der Klasse \myClass{Character}}

Die Klasse \myClass{Character} besitzt folgende Funktionen, die für dich
nützlich sind:

\bgroup
\def\arraystretch{1.2}
\begin{tabularx}{\textwidth}{|p{85mm}|X|}
\hline
\lstinline|public static boolean isDigit(char ch)| & 
Ziffer von ‘0’ bis ‘9’?
\\ \hline
\lstinline|public static boolean isLetter(char ch)| & 
Buchstabe?
\\ \hline
\lstinline|public static boolean isUpperCase(char ch)| & 
Großbuchstabe?
\\ \hline
\lstinline|public static boolean isLowerCase(char ch)| & 
Kleinbuchstabe?
\\ \hline
\end{tabularx}
\egroup
