\section{Animationen erzeugen}

Mit Hilfe der Klasse \myClass{javax.swing.Timer} kann man dafür sorgen, dass
die \lstinline|myPaint()|-Routine in regelmäßigen Abständen immer wieder
aufgerufen wird.

Dies ermöglicht es einfache Animationen zu schreiben, indem man eine Figur, wie
z.B. einen Ball, bei jedem neuen Aufruf von \lstinline|myPaint()| an einer
anderen Position malt. Es entsteht dann der Eindruck einer Bewegung.
Die Klasse \myClass{Timer} besitzt die folgenden Methoden:

\bgroup
\def\arraystretch{1.2}
\begin{tabular}{|l|p{75mm}|}
\hline
\lstinline|Timer(int millisec, ActionListener frame)| &
Über den Konstruktor wird dem Timer mitgeteilt, in welchem zeitlichen Abstand
ein ActionEvent erzeugt werden soll. Der zweite Parameter benennt die Anwendung
die auf den Event reagieren soll. In der Klasse \myClass{HJFrame} ist dies
bereits fertig implementiert, so dass bei jedem so empfangenen ActionEvent das
Fenster neu gezeichnet wird. Als zweiter Parameter wird deshalb von euch
typischerweise \lstinline|this| verwendet.
\\ \hline
\lstinline|void start()| & 
Mit dieser Methode wird der Timer gestartet.
\\ \hline
\lstinline|void stop()| &
Durch Aufruf der Methode \lstinline|stop()| wird der wiederholte Aufruf der
\lstinline|myPaint()|-Methode beendet. Nachdem \lstinline|stop()| aufgerufen
wurde, ist ein erneuter Start möglich.
\\ \hline
\end{tabular}
\egroup

\subsection{Beispiel für eine Animation}

Du findest die Datei \myFile{Animation.java} im Kurs-Repository.

\begin{lstlisting}
æimport java.awt.Color;
import java.awt.EventQueue;
import java.awt.Graphics;æ
import javax.swing.Timer;æ
import hilfe.*;

public class Animation extends HJFrame {
  // globale Variablen
  private static final int WIDTH = 500;
  private static final int HEIGHT = 500;
  private static final Color BACKGROUND = Color.WHITE;
  private static final Color FOREGROUND = Color.BLACK;æ
  private int zaehler = 0;æ

  public Animation(final String title) {
    super(WIDTH, HEIGHT, BACKGROUND, FOREGROUND, title);
    // eigene Initialisierung
æ    Timer timer = new Timer(10, this);
    timer.start();æ
  }
  
  @Override
  public void myPaint(Graphics g) {
    // wird aufgerufen, wenn das Fenster neu gezeichnet wird
æ    zaehler++;æ
    g.drawLine(0, 0, æzaehleræ, æzaehleræ);
    g.drawLine(WIDTH, HEIGHT, WIDTH-æzaehleræ, HEIGHT-æzaehleræ);
    g.drawLine(0, HEIGHT, æzaehleræ, HEIGHT-æzaehleræ);
    g.drawLine(WIDTH, 0, WIDTH-æzaehleræ, æzaehleræ);
  }
  
  public static void main(final String[] args) {
    EventQueue.invokeLater(new Runnable() {
      public void run() {
        try {
          Animation anwendung = new Animation("Animation");
        } catch (Exception e) {
          e.printStackTrace();
        }
      }
    });
  }
}
\end{lstlisting}


\section{Klassen aus dem Package hilfe}

Das Package \myPackage{hilfe} enthält neben der Klasse \myClass{HJFrame} weitere
Vereinfachungen für Programmieraufgaben, die für euch mit den
Standard-Java-Klassen aus dem JRE im Moment noch zu schwer zu programmieren
sind. Damit Klassen aus dem Package \myPackage{hilfe} verwendet werden können,
muss am Anfang einer Java-Datei folgende \lstinline|import|-Anweisung eingefügt
werden:

\begin{lstlisting}
import hilfe.*;
\end{lstlisting}

\subsection{Zeichnen von Dreiecken}

Zum Zeichnen von Dreiecken steht die Klasse \myClass{HZeichnen} zur Verfügung,
die folgende Methoden besitzt:

\begin{compactitem}
\item
\lstinline|static void drawDreieck(Graphics g, int x1, int y1, int x2, int y2, int x3, int y3)| 

\lstinline|drawDreick()| malt ein hohles Dreieck mit den drei Eckpunkten (x1,
y1), (x2, y2) und (x3, y3). Als erster Parameter muss ein Objekt der
Klasse \myClass{Graphics} übergeben werden, da die Methode dieses Objekt zum
Malen braucht.
\item
\lstinline|static void fillDreieck(Graphics g, int x1, int y1, int x2, int y2, int x3, int y3)|

\lstinline|fillDreick()| malt ein gefülltes Dreieck mit den drei Eckpunkten (x1,
y1), (x2, y2) und (x3, y3).
\end{compactitem}

Beispiel für die Verwendung der \lstinline|fillDreieck()|-Methode:

\begin{lstlisting}
HZeichnen.fillDreieck(g, 10, 90, 50, 90, 30, 60);
\end{lstlisting}
