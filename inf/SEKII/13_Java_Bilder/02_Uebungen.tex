\clearpage

\lehead[]{\sf\hspace*{-2.00cm}\textcolor{white}{\colorbox{lightblue}{\parbox[c][0.70cm][b]{1.60cm}{
\makebox[1.60cm][r]{\thechapter}\\ \makebox[1.60cm][r]{ÜBUNG}}}}\hspace{0.17cm}\textcolor{lightblue}{\chaptertitle}}
\rohead[]{\textcolor{lightblue}{\chaptertitle}\sf\hspace*{0.17cm}\textcolor{white}{\colorbox{lightblue}{\parbox[c][0.70cm][b]{1.60cm}{\thechapter\\
ÜBUNG}}}\hspace{-2.00cm}}
%\chead[]{}
\rehead[]{\textcolor{lightblue}{AvHG, Inf, My}}
\lohead[]{\textcolor{lightblue}{AvHG, Inf, My}}

\section{Grafik-Dateien -- Übungen}

\subsection{Aufgabe 1: Himmel mit Vögeln und Wolken}

\begin{compactenum}[a)]
\item Erstelle ein Anwendungsfenster mit einem blauen Hintergrund. Das
 Anwendungsfenster soll eine Breite von 800 Pixeln und eine Höhe von 600 Pixeln
 erhalten.

\item Suche dir aus dem Kursverzeichnis ein Wolkenbild aus, das dir gefällt, und
kopiere es in dein Arbeitsverzeichnis. Lade dieses Wolkenbild im Anwendungsfenster.
Programmiere eine eigene Klasse Wolke, die eine einzelne Wolke an einer
bestimmten Stelle zeichnet.

Im Konstruktor der Klasse werden als Parameter die x- und y-Position der Wolke
und ein Verweis auf das Anwendungsfenster übergeben. Programmiere eine Methode
\lstinline|zeichnen()|, die die Wolke zeichnet.

Erzeuge im Anwendungsfenster mehrere Objekte der Klasse \myClass{Wolke}.

\item Füge in das Anwendungsfenster ein Objekt der Klasse \myClass{Timer} ein
und stelle eine Wiederholungsrate von 150 Millisekunden ein.

\item Kopiere alle Gans-Bilder aus dem Kursverzeichnis in dein
Arbeitsverzeichnis und lade die Bilder im Anwendungsfenster.

Programmiere eine Klasse \myClass{Gans}, die eine einzelne Gans wiederholt von
links nach rechts durch das Fenster fliegen lässt. Die Klasse erhält im
Konstruktor die x-Position, die y-Position, die Geschwindigkeit und einen
Verweis auf das Anwendungsfenster als Parameter. Sie besitzt außerdem eine
Methode \lstinline|zeichnen()|, die die Gans zeichnet und bewegt. Verwende zum
Zeichnen immer abwechselnd eines der vier verschiedenen Gänse-Bilder. Dann wirkt
es so, als würde die Gans mit den Flügeln schlagen.

Erzeuge im Anwendungsfenster mehrere Objekte der Klasse \myClass{Gans}.

\item Kopiere die beiden Vogel-Bilder aus dem Kursverzeichnis in dein
Arbeitsverzeichnis und lade die Vogel- Bilder im Anwendungsfenster.

Programmiere ein Klasse \myClass{Vogel}. Im Konstruktor wird als Parameter nur
ein Verweis auf das Anwendungsfenster übergeben. Die anfängliche x-Position
wird automatisch auf den Wert 500 eingestellt. Die anfängliche y-Position wird
auf 500 gestellt. In der Methode \lstinline|zeichnen()| wird immer abwechselnd
eines der beiden Vogel-Bilder angezeigt, so dass es so wirkt, als würde der
Vogel mit den Flügeln schlagen. Verschiebe außerdem die x-Position bei jedem
Aufruf von \lstinline|zeichnen()| um 15 Pixel nach links und die y-Position um
drei Pixel nach oben. Dann fliegt der Vogel nach links oben. Wenn der Vogel
links aus dem Fenster heraus geflogen ist, soll seine x-Position auf den Wert
800 gesetzt werden, so dass er erneut in das Fenster hinein fliegen kann. Die
y-Position wird nicht verändert.

Erzeuge im Anwendungsfenster ein einzelnes Objekt der Klasse \myClass{Vogel}.
\end{compactenum}


\subsection{Aufgabe 2: Schiffe auf dem Meer}

\begin{compactenum}[a)]
\item Kopiere die drei Schiff-Bilder aus dem Kursverzeichnis in dein
Arbeitsverzeichnis. Erstelle ein Anwendungsfenster mit weißem Hintergrund und
lade die drei Schiff-Bilder.

Programmiere eine Klasse \myClass{Schiff}. Im Konstruktor erhält die Klasse die
x- und y-Position eines Schiffes, die Geschwindigkeit, das Anwendungsfenster
und ein bestimmtes Schiff-Bild als Parameter übergeben. Programmiere eine
\lstinline|zeichnen()|-Methode, die das Schiff mit seinem speziellen Bild
zeichnet (zunächst noch ohne Bewegung).

Erzeuge im Anwendungsfenster mindestens drei verschiedene Objekte der Klasse
\myClass{Schiff}, die unterschiedliche Bilder besitzen. Die Schiffe, die nach
links zeigen, erhalten einen negativen Geschwindigkeitswert.

\item Erweitere die Klasse \myClass{Schiff} so, dass sich die Schiffe je nach
Ausrichtung entweder nach links oder nach rechts bewegen. Wenn ein Schiff aus
dem Fenster hinaus fährt, soll es auf der gegenüberliegenden Seite wieder in
das Fenster hinein gleiten.

\item Erweitere die Klasse \myClass{Schiff} so, dass die Schiffe
Wellenbewegungen ausführen. Dies kann man simulieren in dem man die Schiffe
pixelweise zuerst zehn Pixel nach oben verschiebt und anschließend wieder zehn
Pixel nach unten. Dadurch fahren die Schiffe in einer Zickzacklinie, die wie
eine Wellenbewegung wirkt.
\end{compactenum}

