\chapter{Eclipse als Java-Entwicklungsumgebung}
\newcommand{\chaptertitle}{Eclipse als Java-Entwicklungsumgebung}

\lehead[]{\sf\hspace*{-2.00cm}\textcolor{white}{\colorbox{lightblue}{\makebox[1.60cm][r]{\thechapter}}}\hspace{0.17cm}\textcolor{lightblue}{\chaptertitle}}
\rohead[]{\textcolor{lightblue}{\chaptertitle}\sf\hspace*{0.17cm}\textcolor{white}{\colorbox{lightblue}{\makebox[1.60cm][l]{\thechapter}}}\hspace{-2.00cm}}
%\chead[]{}
\rehead[]{\textcolor{lightblue}{AvHG, Inf, My}}
\lohead[]{\textcolor{lightblue}{AvHG, Inf, My}}


\section{Vorbereitung}

\subsection{Installation des Java-SDKs}

Installiere das aktuelle JDK auf deinem Computer:

\url{http://www.oracle.com/technetwork/java/javase/downloads/index.html}

$\rightarrow$ \myPMI{JDK} $\rightarrow$ \myPMI{Download} $\rightarrow$
\myPMI{Accept License Agreement}

Aus der Liste wählst du die für dein Betriebssystem und Architektur (32bit oder
64bit) passende Datei aus.

\subsection{Anlegen eines Online-Repositories für deine Java-Dateien}

Spätestens wenn mehrere Personen (oft dazu noch an unterschiedlichen Orten)
gemeinsam an einem Software-Projekt arbeiten, macht es keinen Sinn mehr seine
Programm-Quelltexte auf der lokalen Festplatte zu speichern. Vielmehr werden
dann zentrale Versionsverwaltungssysteme – sogenannte Repositories –  benutzt, 
die die Daten online an einer zentralen Stelle bereit halten. Siehe

\url{http://de.wikipedia.org/wiki/Repository}

Wir wollen Subversion benutzen. Beim Anbieter assembla.com kannst du dir hier

\url{https://www.assembla.com/user/one_page_signup?space_type=public}

kostenlos ein öffentliches Subversion-Repository anlegen. Dazu musst du dich
allerdings registrieren (siehe Abbildung \ref{fig:assembla-account-creation}).

\begin{figure}[h]
  \centering
   \includegraphics[width=1.0\textwidth]{./inf/SEKII/01_Vorbereitung/Assembla_Account_Creation.png}
   \caption{Anlegen eines Accounts bei assembla.com}
   \label{fig:assembla-account-creation}
\end{figure}

Bei \myUserInput{Select a Project Configuration} (siehe Abbildung
\ref{fig:assembla-project-configuration}) wählst du \myUserInput{Task and Issue
Management with Integrated Subversion Hosting}.

\begin{figure}[h]
  \centering
   \includegraphics[width=1.0\textwidth]{./inf/SEKII/01_Vorbereitung/Assembla_Project_Configuration.png}
   \caption{Projekt Konfiguration mit Subversion-Repository wählen}
   \label{fig:assembla-project-configuration}
\end{figure}

Du kannst Eclipse auch ohne solch ein Subversion-Repository nutzen, wenn dir
das lieber ist.

\section{Eclipse herunterladen und installieren}

Die jeweils aktuellste stabile Version von Eclipse findest du hier:	

\url{http://www.eclipse.org/downloads/}

Von den dort angebotenen Paketen solltest du die für dein Betriebssystem
passende \glqq Eclipse IDE for Java Developers\grqq\ auswählen. Das
Betriebsystem selbst (Windows, \mbox{MacOS X} oder Linux) wird dabei schon
richtig erkannt -- du musst nur noch zwischen der 32bit- und der 64bit-Version
wählen.

Der Download ist mit deutlich über 100MB recht groß. Dafür ist die Installation
denkbar einfach: Das Heruntergeladene Archiv wird einfach an einen Ort deiner
Wahl (etwa unter \myFile{C:\textbackslash Programme}) entpackt. Du könntest das
Archiv auch direkt auf einen USB-Stick entpacken und damit Eclipse überall
nutzen können. Allerdings wird der Programmstart dadurch verlangsamt, da der
Computer von Festplatte üblicherweise schneller lesen kann als vom USB-Stick.

Wechsele nun in den Ordner, in dem Eclipse entpackt wurde. Etwa in
\myFile{C:\textbackslash Programme\textbackslash eclipse} (evtl. muss das
Programm zum Entpacken des Archivs mit Administrator-Rechten gestartet werden).
Dort kannst du durch einen Rechtsklick auf die Ausführbare eclipse Datei im
Kontextmenü:

\begin{compactitem}

\item Einen Eintrag für Eclipse im Windows-Startmenü erzeugen, indem du aus dem
Kontext-Menü den Eintrag \myPMI{An Startmenü anheften} wählst.

\item Eine Verknüpfung (Icon) auf dem Desktop anlegen, indem du aus dem
Kontext-Menü \myPMI{Senden an\ldots} $\rightarrow$ \myPMI{Desktop} auswählst.

\end{compactitem}

Oder du kannst Eclipse dort auch einfach direkt starten.

\section{Eclipse konfigurieren}

\subsection{Zeichensatz festlegen}

Damit ihr untereinander und mit mir problemlos Dateien austauschen könnt,
solltet ihr als erstes den Zeichensatz festlegen, mit dem Eclipse eure
Textdateien (also auch die Java-Quelltext Dateien) interpretiert. Dies ist im
Besonderen bedeutsam für die Kodierung von Umlauten und sonstigen
Sonderzeichen.

In Eclipse: \myPMI{Window} $\rightarrow$ \myPMI{Preferences} $\rightarrow$
\myPMI{General} $\rightarrow$ \myPMI{Workspace} $\rightarrow$ \myPMI{Text
file encoding} $\rightarrow$ \myPMI{Other} $\rightarrow$ \myPMI{UTF-8}

\subsection{Änderungen durch andere Programme im Workspace überwachen}

Eclipse geht normalerweise davon aus, dass die Dateien im Eclipse-eigenen
Arbeitsverzeichnis (Workspace) nur aus Eclipse selbst heraus angelegt, gelöscht
oder verändert werden.

Das hat zur Folge, dass Eclipse durcheinander kommen kann, wenn du
beispielsweise mit einem externen Programm eine Datei im Workspace veränderst
oder Dateien dort hinein kopierst, löscht oder umbenennst.
Es ist aber leicht, Eclipse so einzustellen, dass externe Änderungen an Dateien
im Workspace erkannt werden:

\myPMI{Window} $\rightarrow$ \myPMI{Preferences} $\rightarrow$
\myPMI{General} $\rightarrow$ \myPMI{Workspace} $\rightarrow$
Häkchen setzen vor \myPMI{Refresh using native hooks or polling}.

\subsection{Subversion Plugin „Subclipse“ installieren und dein eigenes
Repository einbinden} 

Als nächstes solltest du noch eine Erweiterung zu Eclipse installieren um später
mit deinem Sub\-vers\-ion-Repository arbeiten zu können. In Eclipse:
\myPMI{Help} $\rightarrow$ \myPMI{Eclipse Marketplace \ldots} $\rightarrow$ In
das Feld \myPMI{Find:} \myUserInput{subclipse} eintragen und mit \myPMI{Go}
bestätigen. In der Trefferliste (nach kurzer Suche) hinter \myPMI{Subclipse}
auf \myPMI{Install} klicken $\rightarrow$ \myPMI{Next} $\rightarrow$
\myPMI{Next} $\rightarrow$ \myPMI{Accept License Agreement} $\rightarrow$
\myPMI{Finish}.

Damit wird das Subclipse-Plugin installiert. Nach einem anschließend nötigen
Neustart von Eclipse fügst du über \myPMI{Window} $\rightarrow$ \myPMI{Show
View} $\rightarrow$ \myPMI{Other \ldots} $\rightarrow$ \myPMI{SVN}
$\rightarrow$ \myPMI{SVN Repositories} den „SVN Repositories“-View hinzu
(erscheint als zusätzlicher Reiter in dem Teilfenster unten). Ein Rechts-Klick
in diesen View öffnet das dazu gehörige Kontext-Menü. Dort wählst du
\myPMI{New} $\rightarrow$ \myPMI{Repository Location \ldots}. In dem Dialog
gibst du dann als URL \url{https://subversion.assembla.com/svn/DEIN_REPOSITORY}
an (den Namen deines Repositories hast du vorher bei der Registrierung bei
assembla.com selber ausgewählt).

Um mit Eclipse arbeiten zu können musst du zu aller erst ein „Projekt“ anlegen.
Dazu im linken Teil-Fenster (dem sogenannten „Package-Explorer“) ein
Rechts-Klick und dann im Kontext-Menü \myPMI{New} $\rightarrow$ \myPMI{Java
Project} auswählen. Nach der Wahl des Projekt-Namens kann direkt auf
\myPMI{Finish} geklickt werden. 

Mit einem Rechts-Klick auf das neue Projekt (im Package-Explorer nun sichtbar)
wählst du \myPMI{New} $\rightarrow$ \myPMI{Package}. Paket-Namen sollten immer
mit einem kleinen Buchstaben beginnen. Indem du das Projekt „auffaltest“
(Projekt $\rightarrow$ src) siehst du die vorhandenen Pakete des Projektes. Ein
Rechts-Klick auf ein Paket und ein \myPMI{New} $\rightarrow$ \myPMI{Class}
erzeugt einen Dialog, in dem man den Namen der neuen Java-Klasse wählen kann.
Mit einem Rechts-Klick auf ein Projekt im Package-Explorer kann man das Projekt
auch in das zuvor angelegte Subversion-Repository schieben: \myPMI{Team}
$\rightarrow$ \myPMI{Share Project \ldots} $\rightarrow$ \myPMI{SVN}
$\rightarrow$ \myPMI{Next} $\rightarrow$ \myPMI{Next} $\rightarrow$
\myPMI{Finish}. Dies muss nur ein mal für ein Projekt getan werden! 

Dieses Projekt ist nun im Repository angelegt und man kann ab sofort Dateien des
Projekts in dieses Repository sichern (\myPMI{Team} $\rightarrow$
\myPMI{Commit \ldots}) oder auch auf die Dateien aus dem Repository auf die
lokale Festplatte bringen (\myPMI{Team} $\rightarrow$ \myPMI{Update to HEAD})
-- etwa weil du zu Hause gearbeitet hast und nun in der Schule die Dateien auf
dem aktuellen Stand haben möchtest.

\subsection{Das Kurs-Repository einbinden}

Alle Arbeitsblätter und sonstige Dateien werden euch über ein Kurs-Repository
zur Verfügung gestellt. Um auf dieses Kurs-Repository zugreifen zu können,
fügst du mit Rechts-Klick im Repository-View das Kontext-Menü. Dort wählst du
\myPMI{New} $\rightarrow$ \myPMI{Repository Location \ldots}. In dem Dialog
gibst du dann als URL \url{https://subversion.assembla.com/svn/avhg_lk13} an
(der genaue Name eures Kursverzeichnisses wird dir von mir mitgeteilt).

Im Repository View sollte nun nach kurzer Zeit neben deinem eigenen Repository
auch die URL des Kurs-Repositories zu sehen sein. Dieses solltest du
„aufklappen“ (Klick auf den Pfeil vor der URL).

\begin{figure}[h]
  \centering
   \includegraphics[width=1.0\textwidth]{./inf/SEKII/01_Vorbereitung/Eclipse_Repository-View.png}
   \caption{Der Repository-View in Eclipse}
   \label{fig:eclipse-repository-view}
\end{figure}

Anschließend werden mehrere Ordner sichtbar (siehe Abbildung
\ref{fig:eclipse-repository-view}). Mit Rechtsklick auf den Ordner \myFile{LK13}
(bzw. den für euch passenden) bekommst du ein Kontext-Menü. Aus diesem wählst du
den Eintrag \myPMI{Checkout \ldots}. Die folgende Dialog Box kannst direkt mit
\myPMI{Finish} abschließen. Es dauert dann einige Zeit bis das Projekt aus dem
Repository kopiert im Package-Explorer als eigenes Projekt auftaucht.

Wichtig: Im Kurs-Repository hast du nur Lese-, aber keine Schreibrechte.
Folglich kannst du lokale Änderungen auch nicht mit einem Commit in das
Repository sichern. Wenn du Dateien aus dem Kurs-Repository ändern willst musst
du diese deshalb immer zuerst in dein eigenes Repository kopieren. Dort kannst
du sie dann bearbeiten und auch verändert mit einem Commit in dein eigenes
Repository sichern. Falls du versehentlich doch mal etwas im Kurs-Repository
verändert haben solltest (erkennbar am schwarzen Stern vor dem Ordner), dann
kannst du diese Änderungen ganz einfach mit Rechtsklick auf den
Ordner $\rightarrow$ \myPMI{Team} $\rightarrow$ \myPMI{Revert\ldots} rückgängig
machen.


\subsection{Zeilennummern anzeigen lassen}

Beim Programmieren ist es hilfreich, sich im Editor die Zeilennummern anzeigen
zu lassen. Dies stellst du wie folgt ein:

\myPMI{Window} $\rightarrow$ \myPMI{Preferences} $\rightarrow$
\myPMI{General} $\rightarrow$ \myPMI{Editors} $\rightarrow$ \myPMI{Text
Editors} $\rightarrow$ \myPMI{Show line numbers}

\subsection{Automatische Updates} 

Eclipse kann automatisch nach Updates suchen:

\myPMI{Window} $\rightarrow$ \myPMI{Preferences} $\rightarrow$
\myPMI{Install/Update} $\rightarrow$ \myPMI{Automatic Updates} 

In diesem Dialog ein Häkchen Setzen vor \myPMI{Automatically find new updates
and notify me}. Außerdem im Abschnitt \myPMI{Update schedule} des selben
Dialogs auf \myPMI{Look for updates on the following schedule:} $\rightarrow$
\myPMI{Every Saturday} (oder ein beliebiger anderer Wochentag. Im Abschnitt
\myPMI{Download options} wählst du \myPMI{Download new updates automatically
and notify me when ready to install them}. Schließlich den Dialog mit
\myPMI{OK} verlassen.

\subsection{Package „hilfe“ importieren}

Zunächst werdet ihr bei der Java-Programmierung noch häufiger auf Hilfs-Klasse
\myClass{HJFrame} zurück greifen. Um diese nutzen zu können, müsst ihr in eurem
Projekt ein Package \myPackage{hilfe} anlegen (Rechtsklick auf das Projekt
$\rightarrow$ \myPMI{new} $\rightarrow$ \myPMI{package}) und in dieses
anschließend die Dateien  \myFile{HJFrame.java}, \myFile{HZeichnen.java} und
\myFile{EclipseJFrameHilfe.txt} importieren (Rechtsklick auf das Package
$\rightarrow$ \myPMI{Import \ldots} $\rightarrow$ \myPMI{General}
$\rightarrow$ \myPMI{File System} $\rightarrow$ \myPMI{Next} $\rightarrow$
\myPMI{From directory:} (\myPMI{Browse} -- dort das Verzeichnis wählen, in dem
die gewünschten Java-Dateien liegen) $\rightarrow$ \myPMI{Next} $\rightarrow$
im folgenden Dialog alle gewünschten Dateien auswählen $\rightarrow$
\myPMI{Finish}.

Die Java-Dateien sollten jetzt im package \myPackage{hilfe} sichtbar sein.

\subsection{Erzeugung eines Templates (Vorlage) für HJFrame}

\myPMI{Window} $\rightarrow$ \myPMI{Preferences} $\rightarrow$ \myPMI{Java}
$\rightarrow$ \myPMI{Editor} $\rightarrow$ \myPMI{Templates} $\rightarrow$
\myPMI{New}

Im folgenden Dialog als Namen \myUserInput{HJFrame}, als Beschreibung (kann
auch weg gelassen werden) \myUserInput{Template für die Ableitung eigener
Klassen von der HJFrame-Klasse} und in dem großen Textfeld \myPMI{Pattern} den
Text (Copy \& Paste) aus der Datei \myFile{EclipseJFrameHilfe.txt} einfügen.
Anschließend zwei mal mit \myPMI{OK} bestätigen.

Um diese Vorlage zu nutzen genügt es im Editor die Zeichenfolge
\myUserInput{HJFrame} einzutippen (Groß- und Kleinschreibung ist dabei egal) und
durch ein \myUserInput{<Strg>-<Leertaste>} zu bestätigen.


\subsection{Wahl eines externen PDF-Betrachters}

Unter Windows werden PDF-Dateien zumindest in älteren Eclipse-Versionen direkt
geöffnet. Störend dabei ist vor allem, dass Eclipse jedes Mal meint, die
PDF-Datei habe sich verändert und es deshalb beim Schließen der PDF-Ansicht
immer die Rückfrage gibt, ob die Änderungen in der Datei gespeichert werden
sollen. Das ist unsinnig und irritierend und kann im schlimmsten Fall sogar zu
Inkonsistenzen im Repository führen.

Dieses Problem umgehst du, indem du Eclipse anweist, PDF-Dateien nicht selber
anzuzeigen, sondern für diesen Zweck ein externes Programm zu benutzen:

\myPMI{Window} $\rightarrow$ \myPMI{Preferences} $\rightarrow$ \myPMI{General}
$\rightarrow$ \myPMI{Editors} $\rightarrow$ \myPMI{File Associations}

Der folgende Dialog ist in einen oberen und einen unteren Bereich aufgeteilt.
Zunächst klickst du auf \myPMI{Add \ldots} im oberen Bereich und gibst dann
\myUserInput{*.pdf} ein. Anschließend wählst du im unteren Bereich ebenfalls
\myPMI{Add \ldots} und klickst dann im folgenden Dialog auf den Radio-Button für
\myPMI{external Programs}. Aus der folgenden Liste wählst du einen geeigneten
PDF-Betrachter aus. Beispielsweise den Acrobat Reader. Nach Bestätigen mit
\myPMI{OK} ist dieser Konfigurationsschritt abgeschlossen.

Auf die gleiche Art und Weise könntest du auch für andere Dateitypen festlegen,
mit welchen internen oder externen Betrachtern bzw.\ Editoren sie geöffnet
werden sollen.


\subsection{Installation des
MySQL-Java-Connectors}\label{mysql-connector-installation}

Zunächst musst du den Connector herunter laden:

\url{http://dl.dropbox.com/u/31241540/mysql-connector-java-5.1.34-bin.jar}

(Sobald wir im Unterricht bei SQL angekommen sind, wirst du diese Datei auch in
unserem Kurs-Repository finden. Und bis dahin brauchst du dich um die
Einbindung des Connectors auch noch nicht zu kümmern.)

Rechtsklick auf dein Projekt $\rightarrow$ \myPMI{Import \ldots} $\rightarrow$
\myPMI{General} $\rightarrow$ \myPMI{File System} $\rightarrow$ \myPMI{Next}
$\rightarrow$ \myPMI{From directory:} (\myPMI{Browse} -- dort das Verzeichnis
wählen, in dem die JAR-Datei des Connectors liegt) $\rightarrow$ \myPMI{Next}
$\rightarrow$ im folgenden Dialog die gewünschte Datei auswählen $\rightarrow$
\myPMI{Finish}.

Rechtsklick auf deinen Projekt-Ordner im Package-Explorer $\rightarrow$
\myPMI{Build Path} $\rightarrow$ \myPMI{Configure Build Path \ldots}
$\rightarrow$ Wähle den Reiter \myPMI{Libraries} $\rightarrow$ \myPMI{Add JARs
\ldots} $\rightarrow$ wähle \myFile{mysql-connector-java-5.1.34-bin.jar} in
deinem Projekt $\rightarrow$ \myPMI{OK} $\rightarrow$ \myPMI{OK}.

\subsection{Plugin „SQL-Explorer“ installieren und konfigurieren}

\myPMI{Help} $\rightarrow$ \myPMI{Install New Software \ldots} $\rightarrow$
\myPMI{Add} $\rightarrow$ bei Name: \myUserInput{SQL-Explorer} und bei Location:

\url{http://eclipsesql.sourceforge.net/} 

eintragen. Mit \myPMI{OK} bestätigen. Wenn man den nun sichtbaren Eintrag für
\myPMI{SQL Explorer} auffaltet, sieht man möglicherweise mehrere Versionen des
Plugins. Davon die aktuellste auwählen. Mit \myPMI{Next} bestätigen. Nochmals
\myPMI{Next}. Dann \myPMI{I accept the terms of the license agreement}
$\rightarrow$ \myPMI{Finish}. Die im Installationsverlauf erscheinende Warnung
bezüglich der Installation nicht-signierter Inhalte mit \myPMI{OK} quittieren.
Ebenso den Hinweis auf den nötigen Neustart von Eclipse.

Anschließend muss noch die Verbindung zum lokalen MySQL-Server konfiguriert
werden:

\myPMI{Windows} $\rightarrow$ \myPMI{Show View} $\rightarrow$ \myPMI{Other
\ldots} $\rightarrow$ \myPMI{SQL Explorer} $\rightarrow$ \myPMI{Connections}
$\rightarrow$ \myPMI{OK} Im nun verfügbaren Connections-Tab (unten)
$\rightarrow$ Rechtsklick $\rightarrow$ \myPMI{New Connection Profile \ldots}
$\rightarrow$ Name: \myUserInput{MySQL localhost} $\rightarrow$ \myPMI{Add/Edit
Drivers} $\rightarrow$ \myPMI{SQL Explorer} auffalten $\rightarrow$ \myPMI{JDBC
Drivers} $\rightarrow$ Doppelklick auf  \myPMI{MySQL Driver} $\rightarrow$
\myPMI{Extra Class Path} $\rightarrow$ \myPMI{Add JARs \ldots} $\rightarrow$
Den Pfad zu \myFile{mysql-connector-java-5.1.34-bin.jar} (oder neuere Version)
auf der lokalen Platte (vermutlich in einem Unter-Ordner des workspace-Ordners)
auswählen.

Siehe Abbildung \ref{fig:mysql-connector}.

\begin{figure}[h]
  \centering
   \includegraphics[width=1.0\textwidth]{./inf/SEKII/01_Vorbereitung/MySQL-Driver.png}
   \caption{Auswahl des MySQL-Connectors}
   \label{fig:mysql-connector}
\end{figure}

$\rightarrow$ \myPMI{List Drivers} $\rightarrow$ \myPMI{OK}
$\rightarrow$ \myPMI{OK} $\rightarrow$ Driver: \myPMI{MySQL Driver}
$\rightarrow$ Häkchen Setzen bei \myPMI{Auto Logon} und \myPMI{Auto Commit}
$\rightarrow$ Anpassen der URL auf \myUserInput{jdbc:mysql://localhost:3306/}
$\rightarrow$ \myPMI{User:} „root“ $\rightarrow$
\myPMI{Password:} „root“ $\rightarrow$\myPMI{OK}. 

Siehe Abbildung \ref{fig:sql-explorer-connection-profile}.

\begin{figure}[h]
  \centering
   \includegraphics[width=0.7\textwidth]{./inf/SEKII/01_Vorbereitung/SQL-Explorer_Connection-Profile.png}
   \caption{Einstellungsdialog für die Verbindung zum lokalen MySQL-Server}
   \label{fig:sql-explorer-connection-profile}
\end{figure}

Ab sofort werden SQL-Skripte in Eclipse mit dem SQL-Explorer geöffnet. In diesem
kann (einen lokal laufenden MySQL-Server vorausgesetzt)  dann einfach eine
Verbindung zum lokalen MySQL-Server hergestellt werden. Im Drop-Down-Menü,
links von \myPMI{Limit Rows} in der Menüzeile des SQL-Explorer Fensters kann
dazu einfach der Eintrag \myPMI{MySQL localhost/root} ausgewählt werden.

\clearpage

\section{Eclipse kennen lernen}

Für die ersten Schritte in Eclipse empfehle ich dir

\url{http://dl.dropbox.com/u/31241540/JavaBuch_Eclipse.html} 

Dabei handelt es sich um ein Kapitel aus dem „Handbuch der Java-Programmierung“
von Guido Krüger und Heiko Hansen. Dieses ist als HTML-Version kostenlos unter
\url{http://www.javabuch.de/} zu finden oder als richtiges Buch (Addison-Wesley
Verlag) auch zu kaufen.

\begin{minipage}{\textwidth}
Außerdem gibt es auf YouTube jede Menge Video-Tutorials zu Java mit Eclipse.
Beispielsweise auch eine ganze Serie (von brauchbarer Qualität):

\url{http://www.youtube.com/playlist?list=PL71C6DFDDF73835C2}
\end{minipage}
