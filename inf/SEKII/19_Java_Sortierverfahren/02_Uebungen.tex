\clearpage

\lehead[]{\sf\hspace*{-2.00cm}\textcolor{white}{\colorbox{lightblue}{\parbox[c][0.70cm][b]{1.60cm}{
\makebox[1.60cm][r]{\thechapter}\\ \makebox[1.60cm][r]{ÜBUNG}}}}\hspace{0.17cm}\textcolor{lightblue}{\chaptertitle}}
\rohead[]{\textcolor{lightblue}{\chaptertitle}\sf\hspace*{0.17cm}\textcolor{white}{\colorbox{lightblue}{\parbox[c][0.70cm][b]{1.60cm}{\thechapter\\
ÜBUNG}}}\hspace{-2.00cm}}
%\chead[]{}
\rehead[]{\textcolor{lightblue}{AvHG, Inf, My}}
\lohead[]{\textcolor{lightblue}{AvHG, Inf, My}}

\section{Sortierverfahren -- Übungen}

\subsection{Aufgabe 1: Sortieren der Zahlenreihe \glqq 8 3 2 1\grqq}

\subsubsection{Sortieren durch Einfügen (Insertion Sort)}

Zuerst wird überprüft, ob das zweite Element vor das erste geschoben werden
muss.

Nachdem die ersten beiden Elemente in der richtigen Reihenfolge sind, wird das
dritte Element in die bereits sortierte Teilliste eingefügt.

Nachdem die ersten drei Elemente in der richtigen Reihenfolge sind, wird das
vierte Element in die bereits sortierte Teilliste eingefügt.

Und so weiter.

\subsubsection{Sortieren durch Auswählen (Selection Sort)}

Es wird zuerst das kleinste Elemente herausgesucht und mit dem Element an der
ersten Stelle vertauscht. Damit ist das vorderste Element korrekt sortiert.

Anschließend wird aus der unsortierten Restliste das zweit kleinste Element
herausgesucht und mit dem Element an der zweiten Position vertauscht. Damit
sind die beiden vordersten Elemente korrekt sortiert.

Danach wird aus der Restliste das dritt kleinste Element herausgesucht und mit
dem Element an der dritten Position vertauscht, und so weiter.

\subsubsection{Bubble Sort}

Zuerst wird das erste Element mit dem zweiten Element verglichen und
gegebenenfalls vertauscht. Dann wird das zweite Element mit dem dritten Element
verglichen und gegebenenfalls vertauscht. Anschließend wird das dritte Element
mit dem vierten Element verglichen und gegebenenfalls vertauscht, und so weiter
bis das vorletzte Element mit dem letzten Element verglichen und (bei Bedarf)
vertauscht wurde.

Das oben beschriebene Verfahren wird so lange wiederholt, bis bei einem
Zeilendurchgang keine Vertauschungen mehr durchgeführt werden mussten. Dann
befinden sich alle Zahlen in der richtigen Reihenfolge.


\subsection{Aufgabe 2: Namen sortieren}

Du sollst die folgenden Namen alphabetisch ordnen: Meier, Zeller, Müller, Egger

Beginne mit dieser Reihenfolge. Schreibe für alle drei Sortierverfahren die
Reihen nach jedem Sortierschritt auf.


\subsection{Aufgabe 3: Welches Verfahren sortiert am schnellsten?}

Um die Zeit für ein Sortierverfahren exakt zu bestimmen, müsste man die Anzahl
der Vergleiche zwischen zwei Elementen zählen und mit der dafür benötigten
Rechenzeit multiplizieren. Des weiteren müsste die Anzahl der Vertauschungen
von zwei Elementen gezählt werden und wiederum mit der dafür benötigten
Rechenzeit multiplizieren werden. Wenn man beide Werte zusammenaddiert, kann
man die Rechenzeit ziemlich exakt bestimmen.

Wir wollen vereinfacht davon ausgehen, dass ein Verfahren dann besonders
schnell ist, wenn nur wenige „Zeilen“ für die Lösung benötigt werden. Für jede
der unten stehenden Zahlenreihen gibt es nach dieser vereinfachten
Betrachtungsweise ein Sortierverfahren, das optimal ist. Welches ist es jeweils?

\begin{compactenum}[a)]
\item 4  1  2  3
\item 3  2  1  4
\item 2  1  3  4
\end{compactenum}


\subsection{Aufgabe 4: Programmierübung}

\begin{compactenum}[a)]
\item Erstelle ein neues Java-Anwendungsfenster. Erzeuge im Konstruktor der
Anwendung eine Zufallszahl zwischen eins und zehn, die die Länge der
Zahlenliste festlegt. Gib die Zahl zum Test in einem Dialog aus.

\item Erzeuge im Konstruktor ein Array mit Zufallszahlen vom Typ Integer. Das
Array soll die unter a) festgelegte Länge erhalten. Schreibe in die Felder des
Arrays Zufallszahlen zwischen 10 und 99.

\item Schreibe eine Methode, die ein Zahlen-Array in einer bestimmten Zeile auf
dem Bildschirm ausgibt. Die Methode erhält als Parameter das
\myClass{Graphics}-Objekt, das Zahlen-Array und die y-Koordinate der Zeile.
Rufe die Methode in der \lstinline|myPaint()|-Methode des Anwendungsfensters
auf.

\item Schreibe eine Methode, die die größte Zahl der Liste ermittelt. Die
Methode erhält als Parameter das Array und gibt als Rückgabewert die größte
Zahl zurück. Rufe die Methode im Konstruktor auf und gib das Ergebnis der
Methode in einer Messagebox aus.

\item Schreibe eine Methode, die ein Array von Integer-Zahlen als Parameter
erhält und die Zahlen-Liste mit dem Bubble-Sort Verfahren sortiert. Die Methode
gibt keinen Rückgabewert zurück. Ein Rückgabewert ist nicht nötig, da nur die
Speicheradresse der Zahlenliste übergeben wird und das als Parameter übergebene
Array deshalb automatisch mit verändert wird.

Rufe in der \lstinline|myPaint()|-Methode des Anwendungsfensters die
Sortier-Methode für das in Teil b) erzeugte Array auf und gib die sortierte
Liste mit der Ausgabe-Methode aus Teilaufgabe c) aus. Du kannst die
Ausgabe-Methode auch in die Sortiermethode einbauen, um Teilschritte 
des Sortierverfahrens auszutesten.
\end{compactenum}

Zusatzaufgaben:

\begin{compactenum}[a)]
\setcounter{enumi}{5}
\item Programmiere eine Methode, die eine Zahlenliste nach dem Verfahren
Sortieren durch Auswählen sortiert.

\item Programmiere eine Methode, die eine Zahlenliste nach dem Verfahren
Sortieren durch Einfügen sortiert.
\end{compactenum}
