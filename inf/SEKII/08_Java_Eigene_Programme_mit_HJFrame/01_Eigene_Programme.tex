\chapter{Eigene Java Programme}
\renewcommand{\chaptertitle}{Eigene Java Programme}

\lehead[]{\sf\hspace*{-2.00cm}\textcolor{white}{\colorbox{lightblue}{\makebox[1.60cm][r]{\thechapter}}}\hspace{0.17cm}\textcolor{lightblue}{\chaptertitle}}
\rohead[]{\textcolor{lightblue}{\chaptertitle}\sf\hspace*{0.17cm}\textcolor{white}{\colorbox{lightblue}{\makebox[1.60cm][l]{\thechapter}}}\hspace{-2.00cm}}
%\chead[]{}
\rehead[]{\textcolor{lightblue}{AvHG, Inf, My}}
\lohead[]{\textcolor{lightblue}{AvHG, Inf, My}}

\lstset{style=myJava}


\section{Eigene Java-Programme basierend auf HJFrame}

Bisher hast du in Java Aufgaben gelöst, indem du vorgegebene Java-Dateien mit
eigenen Java-Befehlen ergänzt hast. Ab sofort sollst du aber die Programme
komplett selber schreiben. Vereinfacht wird dies zunächst dadurch, dass du
deine Programme von der Klasse \myClass{HJFrame} ableitest und dadurch alle
Eigenschaften und Möglichkeiten von \myClass{HJFrame} erbst. (Was man unter
\glqq ableiten\grqq\ und \glqq erben\grqq\ im Zusammenhang mit
Objektorientierter Programmierung genau versteht, wirst du später noch lernen.)

Wenn du der Anleitung im ersten Kapitel richtig gefolgt bist, hast du bereits
ein Package \myPackage{hilfe} in deinem eigenen Java-Projekt in Eclipse und
hast auch eine Vorlage für die einfache Erstellung eines Programm-Gerüstes,
welches eine neue Java-Klasse basierend auf \myClass{HJFrame} erstellt.

Um diese Vorlage zu nutzen genügt es im Editor die Zeichenfolge
\myUserInput{HJFrame} einzutippen (Groß- und Kleinschreibung ist dabei egal) und
durch ein \myUserInput{<Strg>-<Leertaste>} zu bestätigen.

\subsection{Beispielprogramm}

\begin{lstlisting}[numbers=left, xleftmargin=7mm]
æimport java.awt.Color;
import java.awt.EventQueue;
import java.awt.Graphics;
import hilfe.*;

public class æBeispielæ extends HJFrame {
  final static int WIDTH = 500;
  final static int HEIGHT = 500;
  final static Color BACKGROUND = Color.WHITE;
  final static Color FOREGROUND = Color.BLACK;æ
  String text = "Guten Tag.";æ

  public Beispiel(final String title) {
    super(WIDTH, HEIGHT, BACKGROUND, FOREGROUND, title);
    // eigene Initialisierung
æ    text = text + " Wie geht es dir?";æ
  }

  @Override
  public void myPaint(Graphics g) {
    // wird aufgerufen, wenn das Fenster neu gezeichnet wird 
æ    g.drawString(text, 10, 80);
    g.setColor(Color.RED);
    g.fillOval(100, 100, 200, 200);æ
  }

  public static void main(final String[] args) {
    EventQueue.invokeLater(new Runnable() {
      @Override
      public void run() {
        try {
          Beispiel anwendung = new Beispiel("Beispiel");
        } catch (Exception e) {
          e.printStackTrace();
        }
      }
    });
  }
}
\end{lstlisting}

Anmerkungen zum Beispiel-Programm:

\begin{compactitem}
\item In den Zeilen 1 bis 4 werden Klassen aus anderen Packages, die in der
Datei verwendet werden, aufgeführt. Ein Package enthält eine Gruppe fertiger
Java-Dateien.

Der * bedeutet, dass alle Klassen des Paketes ausgewählt werden.

\item In Zeile sechs wird der Name der Klasse festgelegt. Er muss identisch mit
dem Dateinamen (ohne die \myFile{*.java} Dateiendung) sein. Auch die Groß- und
Kleinschreibung muss dabei beachtet werden!

\item In den Zeilen 13 bis 17 sieht man den Konstruktor. Er hat denselben
Namen wie die Klasse. Der Konstruktor enthält Code, der einmal
beim Start des Programms aufgerufen wird.

\item Zeile 20 bis 25: Die Methode \lstinline|myPaint()| wird vom
System jedes Mal aufgerufen, wenn das Fenster neu gezeichnet
werden muss.

\item Die Methode \lstinline|main()| (ab Zeile 27) wird als erstes aufgerufen,
wenn das Programm gestartet wird.
\end{compactitem}


\section{Malen mit der Klasse Graphics}

Der Code zum Malen des Fensterinhalts wird in die Methode \lstinline|myPaint()|
eingefügt. Das System ruft diese Methode auf, wenn das Fenster neu gezeichnet
werden muss, z.B. weil es vorübergehend durch ein anderes Fenster verdeckt
wurde. 

Vor dem Aufruf von \lstinline|myPaint()| wird der Inhalt des Fensters
automatisch gelöscht.

Damit das Malen für den Programmierer einfacher wird, gibt es die Klasse
\myClass{Graphics}, die einfache Malbefehle beherrscht, wie z.B. das Zeichnen
von Linien oder Rechtecken. Der Programmierer erhält als Parameter der
\lstinline|myPaint()|-Methode ein Objekt der Klasse \myClass{Graphics}
übergeben.

Die nachfolgende Tabelle listet die wichtigsten Methoden der Klasse
\myClass{Graphics} auf. In den Methoden gibt man die Koordinaten von gewünschten
Punkten (Pixel) auf der Zeichenfläche an. Die linke obere Ecke der
Zeichenfläche hat die Position (0,0) des gedachten Koordinatensystems.

\subsection{Methoden von Graphics (Auswahl)}

\begin{tabular}{|l|l|}
\hline
\lstinline|void setFont(Font font)| & Festlegen der Schriftart \\ \hline
\lstinline|void setColor(Color c)| & Festlegen der Vordergrundfarbe \\ \hline
\lstinline|void drawString(String str, int x, int y)| & schreibt einen Text (x,y
= linke untere \\ 
 & Ecke) \\ \hline
\lstinline|void drawLine(int x1, int y1, int x2, int y2)| & malt eine Linie von
(x1, y1) \\ 
& nach (x2, y2)  \\ \hline
\lstinline|void drawRect(int x, int y, int width, int height)| & malt ein
Rechteck (x,y = linke obere \\
\lstinline|void fillRect(int x, int y, int width, int height)| &
Ecke) \lstinline|fillRect()|: ausgefüllt, \\
 &  \lstinline|drawRect()|: hohl \\ \hline
\lstinline|void drawOval(int x, int y, int width, int height)| & malt im Bereich
des angegebenen \\
\lstinline|void fillOval(int x, int y, int width, int height)| & Rechtecks
einen Kreis oder eine Ellipse \\
 & (x,y = linke obere Ecke) \\
 & \lstinline|fillOval()|: ausgefüllt, \\ 
 & \lstinline|drawOval()|: hohl \\ \hline
 \lstinline|void clearRect(int x, int y, int width, int height)| & löscht den angegebenen Bereich \\
 & durch Übermalen mit der Hintergrund- \\
 & farbe (x,y = linke obere Ecke) \\ \hline
\end{tabular}