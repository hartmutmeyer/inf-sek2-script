\clearpage

\section{SQL -- Übung 2: Die Haustierdatenbank wird erweitert}

\subsection{Aufgabe 1: Datenstruktur erweitern}

Die Haustier-Datenbank aus Übung 1 soll erweitert werden. Zu jedem Tier soll
jetzt noch sein Besitzer mit eingetragen werden. Über einen Besitzer sollen
folgende Daten abgespeichert werden:

\begin{compactitem}
  \item \myUserInput{anrede} (Herr, Frau, Firma; Standard-Wert ist „Herr“)
  \item \myUserInput{vorname}
  \item \myUserInput{nachname} (muss eingegeben werden)
  \item \myUserInput{straße} (inklusive Hausnummer)
  \item \myUserInput{plz}
  \item \myUserInput{ort}
  \item \myUserInput{telefonnr}
\end{compactitem}

Selbstverständlich kann ein Besitzer mehrere Tiere haben. Achte darauf, dass
Änderungen z.B. der Adresse eines Besitzers möglichst wenig Datenbank-Änderungen
nach sich ziehen. Wie kann man den Besitzer am geschicktesten in die Datenbank
einbauen?
