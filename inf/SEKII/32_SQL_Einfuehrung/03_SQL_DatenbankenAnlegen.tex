\lstset{style=mySQL}
%\lstset{language=SQL,basicstyle=\ttfamily,commentstyle=\ttfamily,moredelim=*[s][\itshape]{§}{$}}

\section{SQL -- Anlegen von Datenbanken}

\subsection{Groß- und Kleinschreibung}

SQL-Schlüsselwörter sind \textit{nicht} groß- und kleinschreib-gebunden. Ob
Datenbanken und Tabellennamen der Groß- und Kleinschreibung unterliegen hängt
bei MySQL vom Betriebssystem ab! Ihr solltet euch von Anfang an daran gewöhnen,
Groß- und Kleinschreibung bei Namen von Bezeichnern zu unterscheiden!

Empfehlung: Gewöhnt euch an SQL-Schlüsselwörter immer in GROSSBUCHSTABEN und
Bezeichner von Tabellen und Feldern immer in kleinbuchstaben zu schreiben!


\subsection{Datenbank erstellen}

\begin{lstlisting}[language=SQL]
CREATE SCHEMA `\slshape{datenbankname}`;
\end{lstlisting}

Das Ergebnis überprüft man mit dem Befehl

\begin{lstlisting}
SHOW SCHEMA;
\end{lstlisting}


\subsection{Datenbank auswählen}

\begin{lstlisting}
USE `\slshape{datenbankname}`;
\end{lstlisting}

Alle nachfolgenden Anweisungen (z.B. zum Anlegen oder Abfragen von Tabellen)
beziehen sich auf die ausgewählte Datenbank.


\subsection{Den verwendeten Zeichensatz festlegen}

Beim Anlegen einer neuen Datenbank sollte man festlegen, mit welchem Zeichensatz gearbeitet wird. Wir
werden immer den UTF-8 Zeichensatz verwenden:

\begin{lstlisting}
CREATE SCHEMA `\slshape{datenbankname}` DEFAULT CHARACTER SET utf8;
\end{lstlisting}


\subsection{Tabelle erstellen}

\begin{lstlisting}
CREATE TABLE [IF NOT EXISTS] `\slshape{tabellenname}`
(
  `\slshape{Spalten-Definition1}`,
  ...  
  `\slshape{Spalten-DefinitionN}`
);
\end{lstlisting}

Ausdrücke in eckigen Klammern sind optional. Bei Eingabe der Schlüsselwörter
\lstinline{IF NOT EXISTS} wird die Tabelle nur dann angelegt, wenn sie noch
nicht existiert.

Eine \textit{Spalten-Definition} hat folgendes Schema:

\begin{lstlisting}
`\slshape{spaltenname} typ` [NOT NULL] [DEFAULT `\slshape{wert}`] [AUTO_INCREMENT]
\end{lstlisting}

\textit{Typ} gibt den Datentyp der Spalte an (siehe Tabelle
\ref{tab:mysql-datentypen}).

Mit \lstinline{NOT NULL} kann man festlegen, dass der Datenwert der Spalte nicht
leer bleiben darf. „Leere“ Datenwerte werden als \lstinline{NULL} bezeichnet.

Mit \lstinline{DEFAULT} kann man einen Standard-Wert für die Spalte festlegen.

Mit \lstinline{AUTO_INCREMENT} kann man MySQL beauftragen, die Datenwerte
automatisch durch zu nummerieren. Die erste Zeile erhält die Nummer 1. Es darf
nur eine Spalte pro Tabelle auf \lstinline{AUTO_INCREMENT} gesetzt werden. Eine
Spalte darf nur auf \lstinline{AUTO_INCREMENT} gesetzt werden, wenn sie einen
Index besitzt (eine Art Suchregister für besonders schnelle Zugriffe auf die
Einträge). Einen Index kann man manuell anlegen. Wenn man die Spalte zum
\lstinline{PRIMARY KEY} erklärt, wird sie automatisch indiziert:

\begin{lstlisting}
PRIMARY KEY (`\slshape{spaltenname}`);
\end{lstlisting}

Mit \lstinline{PRIMARY KEY} erklärt man eine Spalte zum Primär-Schlüssel, der
jede Zeile eindeutig identifiziert. Der Primär-Schlüssel bestimmt zum Beispiel
die Standard-Sortierreihenfolge der Tabelle.

\begin{table}[h]
\centering
\begin{tabular}{|l|p{11cm}|}
\hline
\bf{Typ} & \bf{Bedeutung}\\
\hline
{\lstinline!INT!} & 4-Byte großer Ganzzahlwert\\
\hline
{\lstinline!FLOAT!} & Fließkommazahl mit einfacher Genauigkeit\\
\hline
{\lstinline!DOUBLE!} & Fließkommazahl mit doppelter Genauigkeit\\
\hline
{\lstinline!VARCHAR(!}\tt{\slshape{maximale Länge}}{\lstinline!)!} &
Zeichenkette. In Klammern gibt man die maximale Anzahl der Zeichen an.
Höchstmöglicher Wert ist 255.\\
\hline
{\lstinline!ENUM(!}\tt{\slshape{Wert1, Wert2, \ldots}}{\lstinline!)!} & 
ermöglicht die Aufzählung einer Werteliste, z.B.:\\
& {\lstinline!ENUM('rot', 'blau', '!}\tt{grün}{\lstinline!')!}\\
\hline
{\lstinline!DATE!} & Datum im Format YYYY-MM-DD\\
\hline
{\lstinline!TIME!} & Uhrzeit im Format HH:MM:SS\\
\hline
{\lstinline!DATETIME!} & Datum und Uhrzeit im Format YYYY-MM-DD HH:MM:SS\\
\hline
\end{tabular}
\caption{Datentypen von MySQL (Auswahl)}
\label{tab:mysql-datentypen}
\end{table}

\subsubsection{Überprüfung der eingegebenen Tabellen}

\begin{lstlisting}
SHOW TABLES;               # Zeigt alle Tabellen der Datenbank an.
DESCRIBE `\slshape{tabellenname}`;     # Zeigt den Aufbau einer Tabelle an.
\end{lstlisting}

\subsection{Datenbanken und Tabellen löschen}

\begin{lstlisting}
DROP SCHEMA [IF EXISTS] `\slshape{datenbankname}`;
DROP TABLE [IF EXISTS] `\slshape{tabellenname}`;
\end{lstlisting}


\subsection{Tabellenstruktur ändern}

Mit der Anweisung \lstinline{ALTER TABLE} \ldots kann man die Struktur einer
existierenden Tabelle verändern. Dabei gibt es zahllose Varianten, z.B.:

\begin{lstlisting}
ALTER TABLE `\slshape{tabellenname}` ADD COLUMN (`\slshape{Spalten-Definition1,Spalten-Definition2,\ldots}`); 
ALTER TABLE `\slshape{tabellenname}` DROP COLUMN `\slshape{spaltenname}`;
\end{lstlisting}


Es können auch mehrere Änderungen in einer \texttt{ALTER}-Anweisung zusammen
gefasst werden. Wie in diesem Beispiel:

\begin{lstlisting}[numbers=left, xleftmargin=7mm]
æDROP SCHEMA IF EXISTS stammbaum;
CREATE SCHEMA stammbaum DEFAULT CHARACTER SET utf8;
USE stammbaum;
CREATE TABLE person (
  vorname VARCHAR(20) NOT NULL,
  nachname VARCHAR(20) NOT NULL,
  geschlecht ENUM('männlich', 'weiblich') DEFAULT 'männlich', 
  anzahl_kinder INT NOT NULL DEFAULT 0
);
INSERT INTO person VALUES
('Hartmut', 'Meyer', 'männlich', 3);æ
ALTER TABLE person
  DROP COLUMN anzahl_kinder,
  ADD COLUMN (
    person_id INT AUTO_INCREMENT,
    anzahl_töchter INT NOT NULL DEFAULT 0,
    anzahl_söhne INT NOT NULL DEFAULT 0
  ),
  ADD PRIMARY KEY (person_id)
;æ
UPDATE person SET anzahl_töchter=1, anzahl_söhne=2
WHERE vorname='Hartmut' AND nachname='Meyer';
\end{lstlisting}