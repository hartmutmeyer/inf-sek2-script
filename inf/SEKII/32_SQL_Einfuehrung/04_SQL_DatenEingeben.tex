\section{SQL -- Daten eingeben}

\subsection{Zeichenketten und Datum- / Zeit-Angaben}

\ldots werden in einfache Anführungsstriche gesetzt, z.B.
\myUserInput{’2004-08-14’}


\subsection{Daten einfügen}

\begin{compactenum}[a)] 
  \item Eingabe von vollständigen Tabellen-Zeilen:

\begin{lstlisting}
INSERT INTO `\slshape{tabellenname}` VALUES
(`\slshape{Wert1, Wert2, \ldots, WertN}`),
`\slshape{\ldots}`
(`\slshape{WertX1, WertX2, \ldots, WertXN}`);
\end{lstlisting}
  
  \item Eingabe von unvollständigen Tabellen-Zeilen:
  
\begin{lstlisting}
INSERT INTO `\slshape{tabellenname}` (`\slshape{spaltenname1, spaltenname2, \ldots}`) VALUES 
(`\slshape{Wert1, Wert2, \ldots}`),
`\slshape{\ldots}`
(`\slshape{WertX1, WertX2, \ldots}`);
\end{lstlisting}
\end{compactenum}


\subsection{Daten der Tabelle ansehen}

Sämtliche Einträge einer Tabelle erhält man mit der Anweisung:

\begin{lstlisting}
SELECT * FROM `\slshape{tabellenname}`;
\end{lstlisting}

\subsection{Daten löschen}

Mit folgender Anweisung löscht man alle Zeilen einer Tabelle:

\begin{lstlisting}
DELETE FROM `\slshape{tabellenname}`;
\end{lstlisting}

Ausgewählte Zeilen löscht man mit der Anweisung:

\begin{lstlisting}
DELETE FROM `\slshape{tabellenname}` WHERE `\slshape{Bedingung}`;
\end{lstlisting}

Beispiel:

\begin{lstlisting}
DELETE FROM adresse WHERE vorname='Michael' AND nachname='Mustermann';
\end{lstlisting}


\subsection{Daten aktualisieren}

\begin{lstlisting}
UPDATE `\slshape{tabellenname}` SET `\slshape{spaltenname1=Wert1, spaltenname2=Wert2, \ldots}`
WHERE `\slshape{bedingung}`;
\end{lstlisting}

Beispiel:

\begin{lstlisting}
UPDATE adresse SET nachname='Meier' WHERE nachname='Mustermann';
\end{lstlisting}

%\pagebreak

\subsection{Vergleichsoperatoren}

Ähnlich wie in Java gibt es auch in SQL alle gängigen Vergeichsoperatoren:

\begin{table}[h]
\centering
\begin{tabular}{|c|c|}
\hline
\textbf{Operator} & \textbf{Bedeutung}\\ \hline
{\lstinline|=|} & gleich \\ \hline
{\lstinline|!=|} oder {\lstinline|<>|} & ungleich \\ \hline
{\lstinline|<|} & kleiner \\ \hline
{\lstinline|>|} & größer \\ \hline
{\lstinline|<=|} & kleiner oder gleich \\ \hline
{\lstinline|>=|} & größer oder gleich \\ \hline
{\lstinline|x IS NULL|} & testet ob x den Wert {\lstinline|NULL|} hat \\ \hline
{\lstinline|AND|} oder {\lstinline|&&|} & und \\ \hline
{\lstinline|OR|} oder {\lstinline!||!} & oder \\ \hline
{\lstinline|XOR|} & entweder oder (exklusives oder) \\ \hline
{\lstinline|NOT|} oder {\lstinline|!|} & nicht \\ \hline
\end{tabular}
\caption{SQL Vergleichsoperatoren}
\label{tab:sql-vergleichsoperatoren}
\end{table}


\subsection{SQL-Anweisungen aus einer Datei einfügen}

Man kann sämtliche SQL-Anweisungen auch in eine Textdatei schreiben. Wenn
einzelne Zeilen in der Datei beim Test nicht ausgeführt werden sollen, fügt man
an den Anfang der Zeile das \glqq Hash\grqq -Zeichen (\#) ein, um aus dem Code
einen Kommentar zu machen.

Wenn MySQL noch nicht gestartet ist, wird die Datei folgendermaßen ausgeführt:

\lstinline{c:\mysql\bin\mysql -u root < c:/Anna/Datei.txt}

Wenn MySQL bereits läuft, führt man die Datei mit diesem Kommando aus:

\lstinline{source c:/Anna/Datei.txt}