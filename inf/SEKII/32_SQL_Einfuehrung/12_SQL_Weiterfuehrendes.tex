\clearpage

\lehead[]{\sf\hspace*{-2.00cm}\textcolor{white}{\colorbox{lightblue}{\makebox[1.60cm][r]{\thechapter}}}\hspace{0.17cm}\textcolor{lightblue}{\chaptertitle}}
\rohead[]{\textcolor{lightblue}{\chaptertitle}\sf\hspace*{0.17cm}\textcolor{white}{\colorbox{lightblue}{\makebox[1.60cm][l]{\thechapter}}}\hspace{-2.00cm}}
%\chead[]{}
\rehead[]{\textcolor{lightblue}{AvHG, Inf, My}}
\lohead[]{\textcolor{lightblue}{AvHG, Inf, My}}

\section[\hspace{1mm} SQL -- Weiterführendes]{SQL -- Weiterführendes}

SQL bietet noch eine ganze Reihe weiterer Sprachelementen, die wir aus
Zeitgründen aber nicht alle behandeln können. Hier nur ein kurzer Überblick über das, was
wir nicht behandelt haben:

\subsection{Ähnlichkeitssuche mit \texttt{LIKE}}

Bisher weißt du, wie du den Wert eines Datenfeldes auf Gleichheit (oder auch
Ungleichheit) mit einem Vergleichswert überprüfen kannst. Etwa in diesem
Beispiel:

\begin{lstlisting}
SELECT * FROM schueler WHERE klasse = '7a';
\end{lstlisting}

Aber was, wenn du nicht nur alle Schüler der 7a, sondern alle Schüler aus allen
siebten Klassen suchst?

\begin{lstlisting}
SELECT * FROM schueler WHERE klasse LIKE '7%';
\end{lstlisting}

Das Prozent-Zeichen ist ein sogenanntes \emph{Wildcard}-Zeichen. \lstinline|%|
steht für beliebig viele (auch kein) beliebige Zeichen.

Als weiteres Wildcard-Zeichen kann der Unterstrich benutzt werden. \lstinline|_|
steht  für genau ein beliebiges Zeichen.

\subsection{\texttt{MAX()}, \texttt{MIN()}, \texttt{AVG()} und \texttt{SUM()}}

Mit \lstinline|MAX(|\emph{ausdruck}\lstinline|)| lässt sich der größte Wert in
Ausdruck ermitteln.

Analoges gilt für \lstinline|MIN(|\emph{ausdruck}\lstinline|)| (liefert den
kleinsten Wert), \lstinline|AVG(|\emph{ausdruck}\lstinline|)| (liefert das
arithmetische Mittel) und \lstinline|SUM(|\emph{ausdruck}\lstinline|)| (liefert
die Summe aller Werte).


\subsection{Mathematische Funktionen}

Man kann in SQL auch rechnen. Wenn in der Tabelle \myUserInput{artikel} in dem
Feld \myUserInput{preis} die jeweiligen Nettopopreise der Artikel gespeichert
sind, so kann man die Bruttopreise direkt in SQL berechnen lassen:

\begin{lstlisting}
SELECT nummer, bezeichnung, preis "Netto-Preis", (preis*1.19) "Brutto-Preis" FROM artikel;
\end{lstlisting}

Neben den Gruzndrechenarten kennt SQL aber auch eine große Anzahl von weiteren
mathematischen Funktionen, wie etwa \lstinline|SIN()|, \lstinline|COS()|,
\lstinline|TAN()|, \lstinline|LN()| usw.

\subsection{Datentyp \texttt{BLOB}}

In Datenbanken können nicht nur Text oder Zahlenwerte abgespeichert werden,
sondern beliebige Daten. Etwa Grafik-Dateien, Musik oder PDF-Dokumente. Für
solche Daten stehen in MySQL spezielle Datentypen zur Verfügung:
Von \myUserInput{BLOB} (maximal 64kB) über \myUserInput{MEDIUMBLOB} (maximal
16MB), bis hin zu \myUserInput{LONGBLOB} (bis zu 4GB).

Um die entsprechenden Daten in der Datenbank zu speichern, oder auch von dort
auszulesen, muss man allerdings ein Programm schreiben (etwa in Java).

Ein Beispiel findest du im Kurs-Repository in
\myUserInput{37\_JavaSQL\_Datenbankzugriffe} im Ordner \myUserInput{blob}.

\subsection{Unterabfragen}

Man kann die Ergebnisse von Abfragen direkt in anderen Abfragen benutzen.

\begin{lstlisting}
SELECT * FROM konto WHERE kontonummer = (SELECT kontonummer FROM konto WHERE konto_id = 1);
\end{lstlisting}

\subsection{Stored Procedures}

Ähnlich wie in Java kann man auch in SQL programmieren. Dazu kann man (wie in
Java) Variablen, Schleifen und Verzweigungen benutzen.
