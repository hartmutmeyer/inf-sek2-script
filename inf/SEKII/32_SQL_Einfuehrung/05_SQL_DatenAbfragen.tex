\section{SQL -- Daten abfragen}

\subsection{SELECT-Anweisung}

Das Auslesen von Informationen aus Tabellen geschieht mit der
\lstinline{SELECT}-Anweisung:

\begin{lstlisting}
SELECT `\slshape{spalte1, spalte2, \ldots}`
FROM `\slshape{tabelle1, tabelle2, \ldots}`
[WHERE `\slshape{Bedingung}`]
[GROUP BY `\slshape{gruppe}`]
[HAVING `\slshape{Gruppen\_Bedingung}`]
[ORDER BY `\slshape{sortierspalten}`];
\end{lstlisting}

Hinter dem Schlüsselwort \lstinline{SELECT} gibt man einen oder mehrere
Spaltennamen an, die ausgelesen werden sollen. Wenn man alle Spalten haben
möchte, schreibt man \lstinline{*}. Wenn man Spalten aus verschiedenen Tabellen
unterscheiden muss, kann man die Spaltenangaben nach folgendem Schema vornehmen:

\texttt{\slshape{tabellenname.spaltenname}}

Hinter dem Schlüsselwort \lstinline{FROM} gibt man die Tabelle oder die Tabellen
an, aus denen man Daten auslesen möchte.

Hinter \lstinline{WHERE} gibt man an, welche Bedingung die ausgewählten Zeilen
erfüllen sollen.


\subsubsection{Zeilen zählen}

Mit der Funktion \lstinline{COUNT(}\texttt{\slshape{Ausdruck}}\lstinline{)} kann
man die Anzahl von Zeilen zählen. Die folgende Anweisung zählt z.B. wie viele
Zeilen sich in der Tabelle \myUserInput{auto} befinden:

\begin{lstlisting}
SELECT COUNT(*) FROM auto;
\end{lstlisting}

\textbf{Achtung:} vor der öffnenden Klammer darf kein Leerzeichen stehen, sonst
gibt es einen Fehler! 


\subsubsection{Duplikate entfernen}

Mit dem Schlüsselwort \lstinline{DISTINCT} kann man aus einem Abfrageergebnis
Duplikate herausschmeißen. Beispiel:

\begin{lstlisting}
SELECT job FROM angestellter;
\end{lstlisting}

ergibt:

\texttt{Buchhalter\\
Buchhalter\\
Programmierer\\
System-Administrator\\
Programmierer\\
Programmierer
}

Wenn man die Mehrfachnennungen im Ergebnis nicht wünscht, dann hilft

\begin{lstlisting}
SELECT DISTINCT job FROM angestellter;
\end{lstlisting}

Dieses ergibt nun:

\texttt{Buchhalter\\
Programmierer\\
System-Administrator
}


\subsubsection{Gruppen bilden}

Mit \lstinline{GROUP BY} kann man die abgerufenen
Zeilen zu Gruppen zusammenfassen. Beispiel:

\begin{lstlisting}
SELECT COUNT(*), job
FROM angestellter
GROUP BY job;
\end{lstlisting}

Diese Anfrage zählt die Anzahl der
Angestellten, die in den einzelnen Jobs tätig
sind, und zwar gruppiert nach dem Job. Die
Ausgabe könnte z.B. so aussehen:

\texttt{2 Buchhalter\\
3 Programmierer\\
1 System-Administrator
}


\subsubsection{Gruppen-Bedingungen}

Mit \lstinline{HAVING} kann man für ausgewählte Gruppen
zusätzliche Bedingungen einführen. Beispiel:

\begin{lstlisting}
SELECT COUNT(*), job
FROM angestellter
GROUP BY job
HAVING COUNT(*)=1;
\end{lstlisting}

Mit dieser Abfrage wählt man die Jobs im
Unternehmen aus, für die jeweils nur ein
Angestellter tätig ist. Das Ergebnis ist nun:

\texttt{1 System-Administrator}


\subsubsection{Suchergebnisse sortieren}

Mit \lstinline{ORDER BY} kann man die Suchergebnisse nach einer oder mehr
Spalten sortieren. Die Sortierung kann entweder aufsteigend (\lstinline{ASC})
oder absteigend (\lstinline{DESC}) sein. Aufsteigend ist die
Standard-Einstellung. Beispiel:

\begin{lstlisting}
SELECT *
FROM angestellter
ORDER BY job ASC, name DESC;
\end{lstlisting}

Die Zeilen werden zunächst nach dem Job in aufsteigender alphabetischer
Reihenfolge sortiert. Falls zwei oder mehr Personen denselben Job haben, werden
sie anschließend noch in umgekehrter alphabetischer Reihenfolge nach dem Namen
sortiert.
