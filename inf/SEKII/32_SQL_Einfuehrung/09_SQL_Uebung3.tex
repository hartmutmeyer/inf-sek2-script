\clearpage

\section{SQL -- Übung 3: Erweiterte Haustierdatenbank}

\subsection{Aufgabe 1: Daten eingeben}

Gib nacheinander die folgenden Daten ein. Benutze dabei gegebenenfalls die
\lstinline{INSERT}-Variante, bei der man nur einzelne Spalten angibt.

\begin{center}
\begin{tabular}{|l|l|}
\hline
\textbf{Besitzer} & \textbf{besitzt die Tiere} \\ 
\hline
Firma, NULL, Zoo Lilliput, Obernstraße 54, 20012, Hamburg, 0721/34 34 12 &
Bello, Lassie\\
\hline
Frau, Sandra, Sandelmann, Kullerweg 12, 28205, Bremen, NULL & 
Daisy (Kanarienvogel)\\
\hline
Herr, Mirco, Sandelmann, Unterstraße 17, 28232, Bremen, 0421/123456 &
Mausi, Blacky, Harald \\
\hline
Herr, Tobias, Winkelmann, NULL, NULL, NULL, NULL &
Daisy (Schildkröte), Hasso\\
\hline
Frau, Sandra, Anderson, NULL, NULL, NULL, NULL &
Maja\\
\hline
\end{tabular}
\end{center}


\subsection{Aufgabe 2: Daten ändern}

Die Adresse von Frau Anderson wurde nachgereicht: Wilhelminenweg 42, 28315,
Bremen. Trage die Änderung in die Tabelle \myUserInput{besitzer} ein.


\subsection{Aufgabe 3: Daten abfragen}

Führe auf der erweiterten Datenbank die folgenden Abfragen durch:

\begin{compactenum}[a)]
\item Zeige ohne Verwendung der \lstinline{WHERE}-Klausel eine Tabelle mit
allen Besitzern (Nachname und Vorname) und Tieren (Name und Tierart) an. Was
für eine Tabelle wird erstellt?
\item Zeige eine Liste mit allen Besitzern und den zu ihnen gehörenden Tieren
an. Es sollen alle Spalten angezeigt werden.
\item Zeige eine Tabelle mit allen Besitzern (Nachname und Vorname) und ihren
Tieren (Name und Tierart) an. Sortiere die Liste in aufsteigender
alphabetischer Reihenfolge zunächst nach dem Nachnamen und dann nach dem
Vornamen.
\item Zeige eine Tabelle mit allen Besitzern (Vor- und Nachname) und ihren
lebendigen Tieren an (Name und Tierart). Sortiere die Liste nach der
Tierart.
\item Zeige alle Tiere von Mirco Sandelmann an (Name, Tierart und „lebendig“).
\item Wähle alle Besitzer von Hunden aus. Zeige Vor- und Nachname der
Besitzer sowie Name, Geburtstag und Todestag des Hundes an.
\item Zeige eine Liste der Besitzer (Vor- und Nachname) und der Anzahl der
Tiere an, die sie besitzen. Sortiere die Liste in umgekehrter alphabetischer
Reihenfolge nach Nachnamen.
\item Zeige alle Besitzer mit Vor- und Nachnamen an, die zwei oder mehr Tiere
besitzen.
\item Zähle die Anzahl der Besitzer, die einen Hund oder eine Katze besitzen.
\end{compactenum}