\clearpage

\lehead[]{\sf\hspace*{-2.00cm}\textcolor{white}{\colorbox{lightblue}{\parbox[c][0.70cm][b]{1.60cm}{
\makebox[1.60cm][r]{\thechapter}\\ \makebox[1.60cm][r]{ÜBUNG}}}}\hspace{0.17cm}\textcolor{lightblue}{\chaptertitle}}
\rohead[]{\textcolor{lightblue}{\chaptertitle}\sf\hspace*{0.17cm}\textcolor{white}{\colorbox{lightblue}{\parbox[c][0.70cm][b]{1.60cm}{\thechapter\\
ÜBUNG}}}\hspace{-2.00cm}}
%\chead[]{}
\rehead[]{\textcolor{lightblue}{AvHG, Inf, My}}
\lohead[]{\textcolor{lightblue}{AvHG, Inf, My}}

\section{SQL -- Übung 1: Eine einfache Haustierdatenbank}

\subsection{Aufgabe 1: Datenbank erzeugen}

Erzeuge eine Haustier-Datenbank. Die Datenbank soll zunächst nur die Tabelle
\myUserInput{tier} mit folgenden Datenfeldern enthalten:

\begin{compactitem}
  \item \myUserInput{name} (muss immer angegeben werden)
  \item \myUserInput{tierart} (muss immer angegeben werden)
  \item \myUserInput{lebendig} (\myUserInput{ja}/\myUserInput{nein};
  Standard-Wert: \myUserInput{ja})
  \item \myUserInput{geschlecht}
  (\myUserInput{männlich}/\myUserInput{weiblich}; Standard-Wert:
  \myUserInput{weiblich})
  \item \myUserInput{geburtstag} (kann unbekannt sein)
  \item \myUserInput{todestag} (kann unbekannt sein oder kann leer sein, weil
  das Tier noch lebt)
\end{compactitem}

Bitte halte beim Anlegen der Tabelle die Reihenfolge der Datenfelder exakt ein.
Welche Datentypen sollten die einzelnen Datenfelder erhalten?


\subsection{Aufgabe 2: Daten eingeben}

Führe die folgenden Anweisungen bitte exakt aus, damit nachher alle für die
weiteren Übungen dieselbe Tabelle haben:

\begin{compactenum}[a)]
\item Gib mit Hilfe der ersten Form der \lstinline{INSERT}-Anweisung die
folgenden Datensätze in die Tabelle \myUserInput{tier} ein:

\begin{compactitem}
  \item Bello, Hund, ja, männlich, 01.05.2003, null
  \item Daisy, Kanarienvogel, nein, weiblich, 06.12.1996, 17.08.2004
  \item Mausi, Katze, ja, weiblich, 17.11.2002, null
\end{compactitem}

\item Gib mit Hilfe der zweiten Form der \lstinline{INSERT}-Anweisung die
folgenden Datensätze ein.

\begin{compactitem}
  \item Daisy, Schildkröte
  \item Lassie, Hund
  \item Maja, Hund
\end{compactitem}

\item Gib mit Hilfe der zweiten Form der \lstinline{INSERT}-Anweisung die
folgenden Datensätze ein:

\begin{compactitem}
  \item Hasso, Hund, männlich
  \item Blacky, Katze, männlich
  \item Harald, Hamster, männlich
\end{compactitem}

\item Nun sollen nachträglich noch einige Datensätze verändert werden:

\begin{compactitem}
  \item Trage für die Schildkröte Daisy den Geburtstag 06.12.2003 ein.
  \item Die Hunde Lassie, Maja und Hasso stammen aus einem Wurf. Trage (mit einer 
  einzigen, möglichst kurzen Anweisung) den Geburtstag 23.04.2004 für die drei
  Hunde ein.
  \item Der Hamster Harald ist verstorben. Trage für Harald (in einer einzigen
  Anweisung) den Geburtstag 29.07.2001, den Todestag 15.09.2003 und ein ’nein’
  für lebendig ein.
\end{compactitem}
\end{compactenum}


\subsection{Aufgabe 3: Daten abfragen}

Nachdem wir die Tabelle \myUserInput{tier} angelegt haben, wollen wir unsere
Mini-Datenbank für Abfragen nutzen. Formuliere geeignete SQL-Anweisungen für die
folgenden Abfragen an die Tabelle \myUserInput{tier}. Speichere die
SQL-Anweisungen in einer Textdatei ab, damit du sie später noch weißt.

\begin{compactenum}[a)]
\item Zeige die gesamte Tabelle tier an.
\item Zeige nur die Spalten \myUserInput{name} und \myUserInput{tierart} an.
\item Liste die Spalten \myUserInput{name} und \myUserInput{tierart} auf.
Sortiere dabei in aufsteigender alphabetischer Reihenfolge nach dem Namen. An
zweiter Stelle soll nach der Tierart sortiert werden (auch aufsteigend).
\item Zeige nur die Spalte \myUserInput{geburtstag} an. Sorge dafür, dass jeder
Wert nur einmal angezeigt wird.
\item Zeige die Spalten \myUserInput{name} und \myUserInput{tierart} für die
Tiere an, die noch leben.
\item Liste die Namen aller Tiere auf, die vor dem Jahr 2004 geboren wurden und
 noch nicht tot sind. Sortiere die Namen in absteigender alphabetischer
 Reihenfolge.
\item Liste die gesamten Spalten aller Tiere auf, die weder Hund noch Katze
sind.
\item Zeige die Namen und die Geburtstage aller Tiere an, für die kein Todestag
 angegeben wurde.
\item Zähle wie viele Tiere nach dem 01.01.2003 geboren wurden.
\item Liste auf, wie viele Tiere es von jeder Tierart gibt. Das Ergebnis soll
eine Tabelle mit einer Spalte \myUserInput{tierart} sein und einer Spalte, die
zu jeder Tierart die Anzahl angibt.
\item Liste alle Tierarten auf, von denen es zwei oder mehr Tiere gibt.
\item Zähle die Anzahl der unterschiedlichen Tierarten, die es gibt.
\end{compactenum}
