\chapter{Abstrakte Klassen und Interfaces}
\renewcommand{\chaptertitle}{Abstrakte Klassen und Interfaces}

\lehead[]{\sf\hspace*{-2.00cm}\textcolor{white}{\colorbox{lightblue}{\makebox[1.60cm][r]{\thechapter}}}\hspace{0.17cm}\textcolor{lightblue}{\chaptertitle}}
\rohead[]{\textcolor{lightblue}{\chaptertitle}\sf\hspace*{0.17cm}\textcolor{white}{\colorbox{lightblue}{\makebox[1.60cm][l]{\thechapter}}}\hspace{-2.00cm}}
%\chead[]{}
\rehead[]{\textcolor{lightblue}{AvHG, Inf, My}}
\lohead[]{\textcolor{lightblue}{AvHG, Inf, My}}

\lstset{style=myJava}

\section{Abstrakte Klassen}

Abstrakte Klassen kennst du bereits aus dem UML-Design. Mit einer abstrakten
Klasse kann man Gemeinsamkeiten einer Reihe verwandter Klassen zusammen fassen
(Variablen, Methoden und Assoziationen). Zum Beispiel ist Fahrzeug eine
abstrakte Oberklasse für die Klassen Auto, Fahrrad und Zug.

Abstrakte Klassen werden in Java durch das Schlüsselwort \lstinline|abstract|
gekennzeichnet. Der Compiler sorgt dafür, dass für eine abstrakte Klasse keine
Objekte erzeugt werden können.

Eine abstrakte Klasse kann nicht nur fertig ausprogrammierte Methoden
enthalten. Es dürfen auch reine „Methodeköpfe“ geschrieben werden. Diese müssen
mit dem Schlüsselwort \lstinline|abstract| gekennzeichnet sein. Jede
abgeleitete Klasse ist verpflichtet, alle aufgelisteten abstrakten Methoden zu
programmieren.

Beispiel:

\begin{lstlisting}
abstract public class Fahrzeug {
    protected int geschwindigkeit = 0;

    public Fahrzeug() {
        geschwindigkeit = 50;
    }

    public abstract void bewegen();
    public abstract void zeichnen(Graphics g);
    
    public int getGeschwindigkeit() {
        return geschwindigkeit;
    }
}
\end{lstlisting}


\section{Interfaces}

Wenn man von einer Klasse aus auf die Methoden einer fremden Klasse zugreifen
will, muss man wissen, welche Methoden die fremde Klasse bereit stellt.

Aber wie kann man ein Programm erstellen, das den Code einer fremden Klasse
verwendet, die zum aktuellen Zeitpunkt noch gar nicht existiert?

Beispiele:

\begin{compactitem}
\item Das Betriebssystem muss die Druckertreiber unterschiedlicher Hersteller
ansprechen können. Nach Fertigstellung des Betriebssystems werden eventuell
neue Drucker hergestellt, für die es auch geeignete Druckertreiber geben muss.

\item Das Java-System soll eine bestimmte Methode eines Anwendungsprogramms
aufrufen, wenn der Benutzer eine Taste gedrückt hat. Das Anwendungsprogramm wird
jedoch erst nach Fertigstellung der Java-Entwicklungsumgebung programmiert.
\end{compactitem}

Lösung:

Es werden eine Reihe von Methodenköpfen definiert, die beide Seiten kennen.
Wenn man nur den Kopf einer Methode hinschreibt, bezeichnet man die Methode als
\emph{abstrakt}. Die Methodenköpfe bilden die \emph{Schnittstelle} (engl.
\emph{Interface}), die beide Seiten kennen müssen, um miteinander kommunizieren
zu können. Zum Beispiel legen Methodenköpfe fest, wie das Betriebssystem einen
Druckertreiber ansprechen muss, und sie sagen dem Hersteller des
Druckertreibers, welche Methoden er programmieren muss.

Formal sieht eine Schnittstelle so ähnlich aus wie eine Klasse. Ein Interface
wird mit dem Schlüsselwort \lstinline|interface| gekennzeichnet. Im Unterschied
zu Klassen dürfen Interfaces nur Methodenköpfe und Konstanten (Variablen mit
dem Schlüsselwort \lstinline|final|) enthalten und sie besitzen keinen
Konstruktor.

Alle aufgelisteten Methoden sind automatisch \lstinline|abstract| und
\lstinline|public|. Deshalb kann man diese beiden Schlüsselwörter bei der
Definition der Methoden auch weglassen.

Beispiel für ein Interface:

\begin{lstlisting}
public interface Drucker {
    public abstract String hersteller();
}
\end{lstlisting}
   
Beispiel für eine Klasse, die dieses Interface implementiert:

\begin{lstlisting}
public class Epson_V12 implements Drucker {

    @Override
    public String hersteller() {
        return "Firma Epson 19.01.2007";
    }
}
\end{lstlisting}

Eine Klasse, die ein Interface implementiert (zum Beispiel der Druckertreiber,
der das Drucker-Interface implementiert), muss alle definierten Methoden in Code
umsetzen (Fachwort: implementieren). Wenn eine Klasse ein Interface umsetzen
möchte, gibt sie dies durch das Schlüsselwort \lstinline|implements| an. Der
Compiler achtet darauf, dass die Klasse alle Methoden des Interfaces besitzt.
Vor allen Methoden, die aus dem Interface „geerbt“ werden, muss zwingend das
Schlüsselwort \lstinline|public| stehen.

Eine Klasse darf gleichzeitig ein oder mehrere Interfaces implementieren und
sich von einer anderen Klasse ableiten. Schema:

\begin{lstlisting}
public class Name extends Superklasse implements Interface1, Interface2
\end{lstlisting}
