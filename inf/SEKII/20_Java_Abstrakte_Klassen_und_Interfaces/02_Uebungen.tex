\clearpage

\lehead[]{\sf\hspace*{-2.00cm}\textcolor{white}{\colorbox{lightblue}{\parbox[c][0.70cm][b]{1.60cm}{
\makebox[1.60cm][r]{\thechapter}\\ \makebox[1.60cm][r]{ÜBUNG}}}}\hspace{0.17cm}\textcolor{lightblue}{\chaptertitle}}
\rohead[]{\textcolor{lightblue}{\chaptertitle}\sf\hspace*{0.17cm}\textcolor{white}{\colorbox{lightblue}{\parbox[c][0.70cm][b]{1.60cm}{\thechapter\\
ÜBUNG}}}\hspace{-2.00cm}}
%\chead[]{}
\rehead[]{\textcolor{lightblue}{AvHG, Inf, My}}
\lohead[]{\textcolor{lightblue}{AvHG, Inf, My}}

\section{Abstrakte Klassen und Interfaces -- Übungen}

\subsection{Aufgabe 1: Ableitung von Abstrakte Klassen}

Es soll in Gruppenarbeit ein Zug mit verschiedenen Waggons entwickelt werden.
Jeder programmiert einen Waggon oder eine Lokomotive (Es sollte eine Lokomotive
mit Blickrichtung nach links und eine mit Blickrichtung nach rechts geben.).
Hinterher kann sich jeder einen oder mehrere Züge aus den vorhandenen Waggons
zusammenbauen und sie über den Bildschirm fahren lassen.

\begin{compactenum}[1.]
\item Waggonbau

Wichtig ist, dass alle Waggons zusammenpassen und dass alle Waggons einheitlich
angesteuert werden können. Deshalb muss zunächst ein Grundentwurf gemacht
werden, der für alle Waggons gilt:
\begin{compactenum}[a)]
\item Es müssen verbindliche Richtlinien für die Maße des Waggons festgelegt
werden. Zum Beispiel müssen sich die Puffer aller Waggons auf derselben Höhe
befinden.
\item Damit jeder jeden Waggon benutzen kann, müssen die Waggons alle dieselben
Methoden bereitstellen. Dazu wird gemeinsam eine abstrakte Klasse \myClass{Waggon}
entwickelt. Jeder leitet seinen eigenen Waggon von der Oberklasse
\myClass{Waggon} ab.
\end{compactenum}

\item Zusammenbau der Züge

Damit nicht jeder Waggon einzeln vom Anwendungsfenster aus gesteuert werden
muss, sollte eine Klasse Zug geschrieben werden, die die Steuerung der
einzelnen Waggons eines Zuges übernimmt.

Auch hier ist es vorteilhaft, wenn verschiedene Züge gleich angesteuert werden
können. Deshalb wird auch für die Züge zunächst eine abstrakte Oberklasse
geschrieben.

\item Steuerung über die Tastatur

Die Züge sollen durch Tastendruck gesteuert werden (diesen AUfgabenteil könnt
ihr erst im Anschluss an das nächste Kapitel lösen). Für jeden Zug wird eine
Taste festgelegt, bei deren Drücken der Zug beschleunigt wird und eine weitere
Taste, durch die der Zug abgebremst wird.
\end{compactenum}


\subsection{Aufgabe 2: Implementierung eines Interfaces}

Gegeben ist ein Interface für geometrische Figuren:

\begin{lstlisting}
public interface Geometrie {
    boolean istEckig();         æ// gibt zurück, ob die Figur eckig ist oder nicht 
æ    int anzahlEcken ();         æ// gibt die Anzahl der Ecken zurück
}
\end{lstlisting}

Die geometrischen Figuren Kreis, Dreieck, Rechteck, Trapez und Raute sollen
dieses Interface implementieren. Suche dir eine Klasse aus, und schreibe für
diese Klasse den Code.
