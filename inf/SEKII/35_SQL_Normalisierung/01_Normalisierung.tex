\chapter{Normalisierung}
\renewcommand{\chaptertitle}{Normalisierung}

\lehead[]{\sf\hspace*{-2.00cm}\textcolor{white}{\colorbox{lightblue}{\makebox[1.60cm][r]{\thechapter}}}\hspace{0.17cm}\textcolor{lightblue}{\chaptertitle}}
\rohead[]{\textcolor{lightblue}{\chaptertitle}\sf\hspace*{0.17cm}\textcolor{white}{\colorbox{lightblue}{\makebox[1.60cm][l]{\thechapter}}}\hspace{-2.00cm}}
%\chead[]{}
\rehead[]{\textcolor{lightblue}{AvHG, Inf, My}}
\lohead[]{\textcolor{lightblue}{AvHG, Inf, My}}

Die Normalisierung ist ein Prozess, der zum Entfernen von Entwurfsfehlern aus
einer Datenbank benutzt wird. 

Ziel dabei ist es, \emph{Datenredundanzen} zu vermeiden. Mit anderen Worten:
Eine Information (etwa die Telefonnummer eines Kunden) sollte nur an einer Stelle in
der Datenbank vorliegen. Anderenfalls ergeben sich folgende Nachteile:

\begin{compactitem}
\item Erhöhter Pflegeaufwand: Sollte sich die Telefonnummer des Kunden ändern,
dann muss man sie an mehreren Stellen in der Datenbank korrigieren.
\item Drohende \emph{Inkonsistenzen} (auch als \emph{Anomalien} bezeichnet):
Wenn es bei einer Änderung versäumt wurde bzw. misslang tatsächlich an allen
Stellen das Datum (in unserem Beispiel: die Telefonnummer des Kunden) korrekt zu
aktualisieren, dann enthält die Datenbank anschließend widersprüchliche
Information. Und das ist fast noch schlimmer als Datenverlust! Den Datenverlust
bemerkt man üblicherweise (und hat dann hoffentlich ein nicht all zu altes
Backup), aber inkonsistente Daten können lange unbemerkt bleiben und in dieser
Zeit schwere Folgeschäden verursachen
\item Erhöhter Ressourcenbedarf: Speicherbedarf (auf jeden Fall) und
Laufzeitverhalten (oft, aber nicht immer) der Datenbank werden durch redundante
Datenhaltung negativ beeinflusst.
\end{compactitem}

Der Normalisierungsprozess besteht aus dem
Aufteilen der Tabellen in immer kleinere Tabellen, die zusammen einen besseren
Entwurf ergeben. Der Datenbank-Entwurf wird nacheinander zuerst in die
1.~Normalform, dann in die 2.~Normalform und zuletzt in die 3.~Normalform
gebracht. Jede Normalform legt eine Reihe von Regeln fest, die eine Datenbank
erfüllen soll. Die höheren Normalformen schließen dabei die Regeln der darunter
liegenden Normalformen mit ein.


\section{Erster Entwurf einer Tabelle}

In den folgenden Tabellen kennzeichnen kursiv gesetzte Attribute die
Zugehörigkeit des Attributs zum Primärschlüssel.

\begin{center}
\begin{tabular}{|l|l|l|l|}
\hline
\textbf{\em cd\_id} & \textbf{album} & \textbf{gründungsjahr} &
\textbf{titelliste} \\ \hline
4711 & Anastacia -- Not That Kind & 1999 & 1. Not That Kind, 
\\
& & & 2. I'm Outta Love, \\
& & & 3. Cowboys \& and Kisses \\ \hline 
4712 & Jefferson Airplane -- Surrealistic Pillow & 1965 & 1. White Rabbit \\
\hline
4713 & Anastacia -- Freak of Nature & 1999 & 1. Freak of Nature, \\
& & &  2. Paid my Dues, \\
& & & 3. Overdue Goodbye \\ \hline
\end{tabular}
\end{center}

Welche Datenbank-Probleme bringt diese Form der Tabelle mit sich?


\section{Erste Normalform}

\textbf{Die Feldinhalte müssen frei von \emph{Wiederholungsgruppen} sein (ein
Datenfeld darf keine Liste von gleichartigen Werten beinhalten).
In jeder Zeile eines Datenfeldes darf also nur ein Wert stehen.}

\textbf{Außerdem müssen die Feldinhalte \emph{atomar} sein. Damit meint man,
dass die Information in den einzelnen Spalten der Tabelle nicht weiter
aufteilbar sein dürfen. Etwa wäre ein Feld \lstinline|adresse|, welches die
Komplette Anschrift enthält \emph{nicht} atomar.}

Unser Beispiel oben enthält sowohl eine Wiederholungsgruppe (das Feld
\lstinline|titel|) als auch ein Feld, welches nicht atomar ist
(\lstinline|album|).

\begin{minipage}{1.0\textwidth}
Geänderte Tabelle (erfüllt die Bedingungen der ersten Normalform):

\begin{center}
\begin{tabular}{|l|l|l|l|l|l|}\hline
\textbf{\em cd\_id} & \textbf{album} & \textbf{interpret} &
\textbf{gründungsjahr} & \textbf{\em track} & \textbf{titel}\\ \hline
4711 & Not That Kind & Anastacia & 1999 & 1 & Not That Kind\\ \hline
4711 & Not That Kind & Anastacia & 1999 & 2 & I’m Outta Love\\ \hline
4711 & Not That Kind & Anastacia & 1999 & 3 & Cowboys \& Kisses\\ \hline
4712 & Surrealistic Pillow & Jefferson Airplane & 1965 & 1 & White Rabbit\\
\hline
4713 & Freak of Nature & Anastacia & 1999 & 1 & Freak of Nature\\ \hline
4713 & Freak of Nature & Anastacia & 1999 & 2 & Paid my Dues\\ \hline
4713 & Freak of Nature & Anastacia & 1999 & 3 & Overdue Goodbye\\ \hline
\end{tabular}
\end{center}
\end{minipage}

(der Primärschlüssel der Tabelle setzt sich zusammen aus den Feldern
\lstinline|cd_id| und \lstinline|track|)


\section{Zweite Normalform}

Zunächst zwei Definitionen:

Ein Attribut einer Tabelle wird als \emph{funktional abhängig} vom
Primärschlüssel bezeichnet, wenn sich aus dem Wert des Primärschlüssels
eindeutig auf den Wert dieses Attributs schließen lässt (Bei Funktionen gilt: zu
jedem x-Wert gehört genau ein y-Wert). Das klingt zwar etwas kompliziert, ist
aber eigentlich eine Selbstverständlichkeit: Wäre diese funktionale
Abhängigkeit nämlich für irgendein Attribut der Tabelle nicht gegeben, dann
wäre der Primärschlüssel kein Primärschlüssel!

Primärschlüssel können sich aber auch aus mehreren Attributen zusammen setzen.
In diesem Fall benutzt man den Begriff der \emph{vollfunktionalen
Abhängigkeit}, wenn ein Attribut tatsächlich erst durch die Kombination aller am
Primärschlüssel beteiligten Attribute eindeutig identifiziert werden kann.

\textbf{Zweite Normalform: Zusätzlich zu den Bedingungen der ersten Normalform
muss jedes nicht dem Schlüssel angehörende Attribut \emph{vollfunktional} vom
Primärschlüssel abhängig sein.}

Der Interpret ist z.B.\ von der \lstinline|cd_id| abhängig, d.h.\ jeder
\lstinline|cd_id| ist eindeutig ein Interpret zugeordnet. Der Track (der ja
Teil des Primärschlüssels der Tabelle ist) wird also gar nicht gebraucht um den
Interpreten zu bestimmen! Gleiches gilt übrigens auch für die Attribute
\lstinline|album| und \lstinline|gründungsjahr|. Diese drei Attribute sind nicht
\emph{vollfunktional} vom zusammengesetzten Primärschlüssel der Tabelle
abhängig.

Geänderte Tabellenstruktur (erfüllt die Bedingungen der zweiten Normalform):

Tabelle \myUserInput{cd}:

\vspace{1mm}

\begin{tabular}{|l|l|l|l|}\hline
\textbf{\em cd\_id} & \textbf{album} & \textbf{interpret} &
\textbf{gründungsjahr}\\ \hline 
4711 & Not That Kind & Anastacia & 1999\\ \hline
4712 & Surrealistic Pillow & Jefferson Airplane & 1965\\ \hline
4713 & Freak of Nature & Anastacia & 1999\\ \hline
\end{tabular}

\vspace{2mm}

Tabelle \myUserInput{stück} (der Primärschlüssel der Tabelle setzt
sich zusammen aus den Feldern \lstinline|cd_id| und \lstinline|track|):

\vspace{1mm}

\begin{tabular}{|l|l|l|}\hline
\textbf{\em cd\_id} & \textbf{\em track} & \textbf{titel}\\ \hline
4711 & 1 & Not That Kind\\ \hline
4711 & 2 & I'm Outta Love\\ \hline
4711 & 3 & Cowboys \& Kisses\\ \hline
4712 & 1 & White Rabbit\\ \hline
4713 & 1 & Freak of Nature\\ \hline
4713 & 2 & Paid my Dues\\ \hline
4713 & 3 & Overdue Goodbyes\\ \hline
\end{tabular}

Welche Schwachstellen bestehen in diesem Datenbankschema noch?


\section{Dritte Normalform}

\textbf{Zusätzlich zu den Bedingungen der ersten und zweiten Normalform, dürfen
die Attribute, die nicht Teil des Primärschlüssels sind, nicht funktional
voneinander abhängig sein.}

Das Gründungsjahr ist vom Interpret, der bisher kein Schlüsselattribut ist,
funktional abhängig. Die Tabellen werden also noch einmal aufgespalten:

\begin{minipage}{0.48\textwidth}
Tabelle \myUserInput{cd}:

\vspace{2mm}

\hspace{0mm}
\begin{tabular}{|l|l|l|}\hline
\textbf{\em cd\_id} & \textbf{album} & \textbf{interpret\_id}\\ \hline 
4711 & Not That Kind       & 311\\ \hline
4712 & Surrealistic Pillow & 312\\ \hline
4713 & Freak of Nature     & 311\\ \hline
\end{tabular}
\end{minipage}
\begin{minipage}{0.52\textwidth}
Tabelle \myUserInput{interpret}:

\vspace{2mm}

\begin{tabular}{|l|l|l|}\hline
\textbf{\em interpret\_id} & \textbf{name} & \textbf{gründungsjahr}\\ \hline
311 & Anastacia          & 1999\\ \hline
312 & Jefferson Airplane & 1965\\ \hline
\end{tabular}
\end{minipage}

\vspace{4mm}

\begin{minipage}{1.0\textwidth}
Tabelle \myUserInput{stück}:

\vspace{2mm}

\begin{tabular}{|l|l|l|}\hline
\textbf{\em cd\_id} & \textbf{\em track} & \textbf{titel}\\ \hline
4711 & 1 & Not That Kind\\ \hline
4711 & 2 & I'm Outta Love\\ \hline
4711 & 3 & Cowboys \& Kisses\\ \hline
4712 & 1 & White Rabbit\\ \hline
4713 & 1 & Freak of Nature\\ \hline
4713 & 2 & Paid my Dues\\ \hline
4713 & 3 & Overdue Goodbyes\\ \hline
\end{tabular}
\end{minipage}


\section{Höhere Normalformen}

Es existieren noch höhere Normalformen (vierte, fünfte, usw.), die jedoch
hauptsächlich akademisch interessant sind. Sie werden in der Praxis nicht
eingesetzt, da eine weitere Normalisierung einen zu hohen Aufwand bedeuten
würde. Die dritte Normalform genügt zur Beseitigung typischer Datenredundanzen.


\section{Lesetipp}

\url{http://de.wikipedia.org/wiki/Normalisierung_(Datenbank)}