Die folgende Tabelle soll das Fortbildungssystem einer multinationalen Firma
beschreiben, die Fortbildungszentren in verschiedenen Städten besitzt. Ein
kursiv gesetztes Attribut kennzeichnet einen Primärschlüssel. Zum
konzeptionellen Verständnis: Ein Dozent bietet immer nur Kurse an einem Ort an. Das ist nicht
selbstverständlich, soll in diesem Fall aber so sein.

\begin{tabular}{|r|r|r|r|r|r|r|r|}\hline
\textbf{dozent} & \textbf{dozent} & \textbf{schule} &
\textbf{schule} & \textbf{hörer} & \textbf{hörer} &
\textbf{kurs} & \textbf{kurs}\\
\textbf{\_id} & \textbf{\_name} & \textbf{\_id} &
\textbf{\_ort} & \textbf{\_id} & \textbf{\_name} &
 & \textbf{\_tage}\\
\hline
4711 & Alfred Meier & 001 & München & 8001 & Erwin Huber & Cobol & 10\\ \hline
4750 & Stefan Schmidt & 003 & Wien & 8001, 8432 & Erwin Huber, & APL,
Basic & 15, 5\\
& & & & & Gerd Müller & & \\ \hline
5000 & Wolfgang Adam & 003 & Wien & 8432 & Müller & Cobol & 10\\ \hline
6000 & Eliza Wang & 004 & Berlin & 8001, 8432 & Erwin Huber,& Cobol, APL
& 10, 15\\
& & & & & Gerd Müller & & \\ \hline
\end{tabular}

Welche Datenbank-Probleme bringt diese Form der Tabelle mit sich?


\section{Erste Normalform}

\textbf{Die Feldinhalte müssen frei von \emph{Wiederholungsgruppen} sein (ein
Datenfeld darf keine Liste von gleichartigen Werten beinhalten).
In jeder Zeile eines Datenfeldes darf also nur ein Wert stehen.}

\textbf{Außerdem müssen die Feldinhalte \emph{atomar} sein. Damit meint man,
dass die Information in den einzelnen Spalten der Tabelle nicht weiter
aufteilbar sein dürfen. Etwa wäre ein Feld \lstinline|adresse|, welches die
Komplette Anschrift enthält \emph{nicht} atomar.}

Unser Beispiel oben enthält sowohl eine Wiederholungsgruppe (die Felder
\lstinline|hörer_id|, \lstinline|hörer_name|, \lstinline|kurs| und
\lstinline|kurs_tage|) als auch zwei Felder, welche nicht atomar sind
(\lstinline|dozent_name| und \lstinline|hörer_name|).

\begin{minipage}{1.0\textwidth}
Geänderte Tabelle (erfüllt die Bedingungen der ersten Normalform):

\begin{center}
\begin{tabular}{|r|r|r|r|r|r|r|r|r|r|}\hline
\textbf{dozent} & \textbf{dozent\_} & \textbf{dozent\_} & \textbf{schule} &
\textbf{schule} & \textbf{hörer} & \textbf{hörer\_} & \textbf{hörer\_} &
\textbf{kurs} & \textbf{kurs}\\
\textbf{\_id} & \textbf{vorname} & \textbf{nachname} & \textbf{\_id} &
\textbf{\_ort} & \textbf{\_id}  & \textbf{vorname} & \textbf{nachname}
& & \textbf{\_tage}\\
\hline
4711 & Alfred & Meier & 001 & München & 8001 & Erwin & Huber & Cobol & 10\\
\hline 
4750 & Stefan & Schmidt & 003 & Wien & 8001 & Erwin & Huber & APL & 15\\ 
\hline
4750 & Stefan & Schmidt & 003 & Wien & 8432 & Gerd & Müller & Basic & 5\\ 
\hline 
5000 & Wolfgang & Adam & 003 & Wien & 8432 & Gerd & Müller & Cobol & 10\\ 
\hline
6000 & Eliza & Wang & 004 & Berlin & 8001 & Erwin & Huber & Cobol & 10\\ 
\hline
6000 & Eliza & Wang & 004 & Berlin & 8432 & Gerd & Müller & APL & 15\\ 
\hline
\end{tabular}
\end{center}
\end{minipage}

\section{Zweite Normalform}

\textbf{Zusätzlich zu den Bedingungen der ersten Normalform muss nun jedes
nicht dem Schlüssel angehörende Attribut funktional vom Primärschlüssel abhängig
sein.}

(Bei Funktionen gilt: zu jedem x-Wert gehört genau ein y-Wert.) 

Der Dozentenname ist z.B.\ von der Dozentennummer abhängig, d.h.\ jeder
Dozentennummer ist eindeutig ein Dozentenname zugeordnet. Da ein Dozent in der
Regel mehrere Zuhörer besitzt, wird der Name eines (Zu)Hörers jedoch nicht
eindeutig von der Dozentennummer bestimmt. Die Tabelle muss deshalb in mehrere
kleine Tabellen unterteilt werden, die jeweils einen eigenen Primärschlüssel
erhalten:

\begin{minipage}{0.65\textwidth}
Tabelle \myUserInput{dozent}:

\vspace{2mm}

\hspace{0mm}
\begin{tabular}{|r|r|r|r|r|}\hline
\textbf{\em dozent\_id} & \textbf{vorname} & \textbf{nachname}
& \textbf{schule\_id} & \textbf{schule\_ort}\\ \hline 
4711 & Alfred   & Meier   & 001 & München\\ \hline 
4750 & Stefan   & Schmidt & 003 & Wien   \\ \hline
5000 & Wolfgang & Adam    & 003 & Wien   \\ \hline
6000 & Eliza    & Wang    & 004 & Berlin \\ \hline
\end{tabular}
\end{minipage}
\begin{minipage}{0.30\textwidth}
Tabelle \myUserInput{hörer}:

\vspace{2mm}

\begin{tabular}{|r|r|r|}\hline
\textbf{\em hörer\_id} & \textbf{vorname} & \textbf{nachname}\\ \hline
8001 & Erwin & Huber  \\ \hline
8432 & Gerd & Müller \\ \hline
\end{tabular}
\end{minipage}

\vspace{2mm}

Tabelle \myUserInput{kursbelegung} (der Primärschlüssel dieser Tabelle setzt
sich zusammen aus den Felder \lstinline|dozent_id| und \lstinline|hörer_id|):

\hspace{0mm}
\begin{tabular}{|r|r|r|r|}\hline
\textbf{\em dozent\_id} & \textbf{\em hörer\_id} & \textbf{kurs} &
\textbf{kurs\_tage}\\ \hline 4711 & 8001 & Cobol & 10\\ \hline
4750 & 8001 & APL   & 15\\ \hline
4750 & 8432 & Basic &  5\\ \hline
5000 & 8432 & Cobol & 10\\ \hline
6000 & 8001 & Cobol & 10\\ \hline
6000 & 8432 & APL   & 15\\ \hline
\end{tabular}

Welche Schwachstellen bestehen in diesem Datenbankschema noch?


\section{Dritte Normalform}

\textbf{Zusätzlich zu den Bedingungen der ersten und zweiten Normalform, dürfen
die Attribute, die nicht Teil des Primärschlüssels sind, nicht funktional
voneinander abhängig sein.}

Der Schulort ist zum Beispiel von der Schulnummer, die bisher kein
Schlüsselattribut ist, funktional abhängig. Die Tabellen werden also noch
einmal aufgespalten:

\begin{minipage}{0.7\textwidth}
Tabelle \myUserInput{dozent}:

\vspace{2mm}

\hspace{0mm}
\begin{tabular}{|r|r|r|r|}\hline
\textbf{\em dozent\_id} & \textbf{vorname} & \textbf{nachname} &
\textbf{schule\_id}\\ \hline 
4711 & Alfred & Meier   & 001\\ \hline
4750 & Stefan & Schmidt & 003\\ \hline
5000 & Wolfgang & Adam    & 003\\ \hline
6000 & Eliza & Wang    & 004\\ \hline
\end{tabular}
\end{minipage}
\begin{minipage}{0.20\textwidth}
Tabelle \myUserInput{schule}:

\vspace{2mm}

\begin{tabular}{|r|r|}\hline
\textbf{\em schule\_id} & \textbf{ort}\\ \hline
001 & München\\ \hline
003 & Wien   \\ \hline
004 & Berlin \\ \hline
\end{tabular}
\end{minipage}

\vspace{4mm}

\begin{minipage}{0.4\textwidth}
Tabelle \myUserInput{hörer}:

\vspace{2mm}

\begin{tabular}{|r|r|r|}\hline
\textbf{\em hörer\_id} & \textbf{vorname} & \textbf{nachname}\\ \hline
8001 & Erwin & Huber  \\ \hline
8432 & Gerd & Müller \\ \hline
\end{tabular}
\end{minipage}
\begin{minipage}{0.40\textwidth}
Tabelle \myUserInput{kursbelegung}:

\vspace{2mm}

\begin{tabular}{|r|r|r|}\hline
\textbf{\em dozent\_id} & \textbf{\em hörer\_id} & \textbf{kurs} \\ \hline 
4711 & 8001 & Cobol \\ \hline
4750 & 8001 & APL   \\ \hline
4750 & 8432 & Basic \\ \hline
5000 & 8432 & Cobol \\ \hline
6000 & 8001 & Cobol \\ \hline
6000 & 8432 & APL   \\ \hline
\end{tabular}
\end{minipage}
\begin{minipage}{0.30\textwidth}
Tabelle \myUserInput{kurs}:

\vspace{2mm}

\begin{tabular}{|r|r|}\hline
\textbf{\em kurs} & \textbf{tage}\\ \hline
Cobol & 10 \\ \hline
APL   & 15 \\ \hline
Basic &  5 \\ \hline
\end{tabular}
\end{minipage}